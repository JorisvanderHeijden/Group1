
% Default to the notebook output style

    


% Inherit from the specified cell style.




    
\documentclass[11pt]{article}

    
    
    \usepackage[T1]{fontenc}
    % Nicer default font (+ math font) than Computer Modern for most use cases
    \usepackage{mathpazo}

    % Basic figure setup, for now with no caption control since it's done
    % automatically by Pandoc (which extracts ![](path) syntax from Markdown).
    \usepackage{graphicx}
    % We will generate all images so they have a width \maxwidth. This means
    % that they will get their normal width if they fit onto the page, but
    % are scaled down if they would overflow the margins.
    \makeatletter
    \def\maxwidth{\ifdim\Gin@nat@width>\linewidth\linewidth
    \else\Gin@nat@width\fi}
    \makeatother
    \let\Oldincludegraphics\includegraphics
    % Set max figure width to be 80% of text width, for now hardcoded.
    \renewcommand{\includegraphics}[1]{\Oldincludegraphics[width=.8\maxwidth]{#1}}
    % Ensure that by default, figures have no caption (until we provide a
    % proper Figure object with a Caption API and a way to capture that
    % in the conversion process - todo).
    \usepackage{caption}
    \DeclareCaptionLabelFormat{nolabel}{}
    \captionsetup{labelformat=nolabel}

    \usepackage{adjustbox} % Used to constrain images to a maximum size 
    \usepackage{xcolor} % Allow colors to be defined
    \usepackage{enumerate} % Needed for markdown enumerations to work
    \usepackage{geometry} % Used to adjust the document margins
    \usepackage{amsmath} % Equations
    \usepackage{amssymb} % Equations
    \usepackage{textcomp} % defines textquotesingle
    % Hack from http://tex.stackexchange.com/a/47451/13684:
    \AtBeginDocument{%
        \def\PYZsq{\textquotesingle}% Upright quotes in Pygmentized code
    }
    \usepackage{upquote} % Upright quotes for verbatim code
    \usepackage{eurosym} % defines \euro
    \usepackage[mathletters]{ucs} % Extended unicode (utf-8) support
    \usepackage[utf8x]{inputenc} % Allow utf-8 characters in the tex document
    \usepackage{fancyvrb} % verbatim replacement that allows latex
    \usepackage{grffile} % extends the file name processing of package graphics 
                         % to support a larger range 
    % The hyperref package gives us a pdf with properly built
    % internal navigation ('pdf bookmarks' for the table of contents,
    % internal cross-reference links, web links for URLs, etc.)
    \usepackage{hyperref}
    \usepackage{longtable} % longtable support required by pandoc >1.10
    \usepackage{booktabs}  % table support for pandoc > 1.12.2
    \usepackage[inline]{enumitem} % IRkernel/repr support (it uses the enumerate* environment)
    \usepackage[normalem]{ulem} % ulem is needed to support strikethroughs (\sout)
                                % normalem makes italics be italics, not underlines
    

    
    
    % Colors for the hyperref package
    \definecolor{urlcolor}{rgb}{0,.145,.698}
    \definecolor{linkcolor}{rgb}{.71,0.21,0.01}
    \definecolor{citecolor}{rgb}{.12,.54,.11}

    % ANSI colors
    \definecolor{ansi-black}{HTML}{3E424D}
    \definecolor{ansi-black-intense}{HTML}{282C36}
    \definecolor{ansi-red}{HTML}{E75C58}
    \definecolor{ansi-red-intense}{HTML}{B22B31}
    \definecolor{ansi-green}{HTML}{00A250}
    \definecolor{ansi-green-intense}{HTML}{007427}
    \definecolor{ansi-yellow}{HTML}{DDB62B}
    \definecolor{ansi-yellow-intense}{HTML}{B27D12}
    \definecolor{ansi-blue}{HTML}{208FFB}
    \definecolor{ansi-blue-intense}{HTML}{0065CA}
    \definecolor{ansi-magenta}{HTML}{D160C4}
    \definecolor{ansi-magenta-intense}{HTML}{A03196}
    \definecolor{ansi-cyan}{HTML}{60C6C8}
    \definecolor{ansi-cyan-intense}{HTML}{258F8F}
    \definecolor{ansi-white}{HTML}{C5C1B4}
    \definecolor{ansi-white-intense}{HTML}{A1A6B2}

    % commands and environments needed by pandoc snippets
    % extracted from the output of `pandoc -s`
    \providecommand{\tightlist}{%
      \setlength{\itemsep}{0pt}\setlength{\parskip}{0pt}}
    \DefineVerbatimEnvironment{Highlighting}{Verbatim}{commandchars=\\\{\}}
    % Add ',fontsize=\small' for more characters per line
    \newenvironment{Shaded}{}{}
    \newcommand{\KeywordTok}[1]{\textcolor[rgb]{0.00,0.44,0.13}{\textbf{{#1}}}}
    \newcommand{\DataTypeTok}[1]{\textcolor[rgb]{0.56,0.13,0.00}{{#1}}}
    \newcommand{\DecValTok}[1]{\textcolor[rgb]{0.25,0.63,0.44}{{#1}}}
    \newcommand{\BaseNTok}[1]{\textcolor[rgb]{0.25,0.63,0.44}{{#1}}}
    \newcommand{\FloatTok}[1]{\textcolor[rgb]{0.25,0.63,0.44}{{#1}}}
    \newcommand{\CharTok}[1]{\textcolor[rgb]{0.25,0.44,0.63}{{#1}}}
    \newcommand{\StringTok}[1]{\textcolor[rgb]{0.25,0.44,0.63}{{#1}}}
    \newcommand{\CommentTok}[1]{\textcolor[rgb]{0.38,0.63,0.69}{\textit{{#1}}}}
    \newcommand{\OtherTok}[1]{\textcolor[rgb]{0.00,0.44,0.13}{{#1}}}
    \newcommand{\AlertTok}[1]{\textcolor[rgb]{1.00,0.00,0.00}{\textbf{{#1}}}}
    \newcommand{\FunctionTok}[1]{\textcolor[rgb]{0.02,0.16,0.49}{{#1}}}
    \newcommand{\RegionMarkerTok}[1]{{#1}}
    \newcommand{\ErrorTok}[1]{\textcolor[rgb]{1.00,0.00,0.00}{\textbf{{#1}}}}
    \newcommand{\NormalTok}[1]{{#1}}
    
    % Additional commands for more recent versions of Pandoc
    \newcommand{\ConstantTok}[1]{\textcolor[rgb]{0.53,0.00,0.00}{{#1}}}
    \newcommand{\SpecialCharTok}[1]{\textcolor[rgb]{0.25,0.44,0.63}{{#1}}}
    \newcommand{\VerbatimStringTok}[1]{\textcolor[rgb]{0.25,0.44,0.63}{{#1}}}
    \newcommand{\SpecialStringTok}[1]{\textcolor[rgb]{0.73,0.40,0.53}{{#1}}}
    \newcommand{\ImportTok}[1]{{#1}}
    \newcommand{\DocumentationTok}[1]{\textcolor[rgb]{0.73,0.13,0.13}{\textit{{#1}}}}
    \newcommand{\AnnotationTok}[1]{\textcolor[rgb]{0.38,0.63,0.69}{\textbf{\textit{{#1}}}}}
    \newcommand{\CommentVarTok}[1]{\textcolor[rgb]{0.38,0.63,0.69}{\textbf{\textit{{#1}}}}}
    \newcommand{\VariableTok}[1]{\textcolor[rgb]{0.10,0.09,0.49}{{#1}}}
    \newcommand{\ControlFlowTok}[1]{\textcolor[rgb]{0.00,0.44,0.13}{\textbf{{#1}}}}
    \newcommand{\OperatorTok}[1]{\textcolor[rgb]{0.40,0.40,0.40}{{#1}}}
    \newcommand{\BuiltInTok}[1]{{#1}}
    \newcommand{\ExtensionTok}[1]{{#1}}
    \newcommand{\PreprocessorTok}[1]{\textcolor[rgb]{0.74,0.48,0.00}{{#1}}}
    \newcommand{\AttributeTok}[1]{\textcolor[rgb]{0.49,0.56,0.16}{{#1}}}
    \newcommand{\InformationTok}[1]{\textcolor[rgb]{0.38,0.63,0.69}{\textbf{\textit{{#1}}}}}
    \newcommand{\WarningTok}[1]{\textcolor[rgb]{0.38,0.63,0.69}{\textbf{\textit{{#1}}}}}
    
    
    % Define a nice break command that doesn't care if a line doesn't already
    % exist.
    \def\br{\hspace*{\fill} \\* }
    % Math Jax compatability definitions
    \def\gt{>}
    \def\lt{<}
    % Document parameters
    \title{Assignment 3}
    
    
    

    % Pygments definitions
    
\makeatletter
\def\PY@reset{\let\PY@it=\relax \let\PY@bf=\relax%
    \let\PY@ul=\relax \let\PY@tc=\relax%
    \let\PY@bc=\relax \let\PY@ff=\relax}
\def\PY@tok#1{\csname PY@tok@#1\endcsname}
\def\PY@toks#1+{\ifx\relax#1\empty\else%
    \PY@tok{#1}\expandafter\PY@toks\fi}
\def\PY@do#1{\PY@bc{\PY@tc{\PY@ul{%
    \PY@it{\PY@bf{\PY@ff{#1}}}}}}}
\def\PY#1#2{\PY@reset\PY@toks#1+\relax+\PY@do{#2}}

\expandafter\def\csname PY@tok@w\endcsname{\def\PY@tc##1{\textcolor[rgb]{0.73,0.73,0.73}{##1}}}
\expandafter\def\csname PY@tok@c\endcsname{\let\PY@it=\textit\def\PY@tc##1{\textcolor[rgb]{0.25,0.50,0.50}{##1}}}
\expandafter\def\csname PY@tok@cp\endcsname{\def\PY@tc##1{\textcolor[rgb]{0.74,0.48,0.00}{##1}}}
\expandafter\def\csname PY@tok@k\endcsname{\let\PY@bf=\textbf\def\PY@tc##1{\textcolor[rgb]{0.00,0.50,0.00}{##1}}}
\expandafter\def\csname PY@tok@kp\endcsname{\def\PY@tc##1{\textcolor[rgb]{0.00,0.50,0.00}{##1}}}
\expandafter\def\csname PY@tok@kt\endcsname{\def\PY@tc##1{\textcolor[rgb]{0.69,0.00,0.25}{##1}}}
\expandafter\def\csname PY@tok@o\endcsname{\def\PY@tc##1{\textcolor[rgb]{0.40,0.40,0.40}{##1}}}
\expandafter\def\csname PY@tok@ow\endcsname{\let\PY@bf=\textbf\def\PY@tc##1{\textcolor[rgb]{0.67,0.13,1.00}{##1}}}
\expandafter\def\csname PY@tok@nb\endcsname{\def\PY@tc##1{\textcolor[rgb]{0.00,0.50,0.00}{##1}}}
\expandafter\def\csname PY@tok@nf\endcsname{\def\PY@tc##1{\textcolor[rgb]{0.00,0.00,1.00}{##1}}}
\expandafter\def\csname PY@tok@nc\endcsname{\let\PY@bf=\textbf\def\PY@tc##1{\textcolor[rgb]{0.00,0.00,1.00}{##1}}}
\expandafter\def\csname PY@tok@nn\endcsname{\let\PY@bf=\textbf\def\PY@tc##1{\textcolor[rgb]{0.00,0.00,1.00}{##1}}}
\expandafter\def\csname PY@tok@ne\endcsname{\let\PY@bf=\textbf\def\PY@tc##1{\textcolor[rgb]{0.82,0.25,0.23}{##1}}}
\expandafter\def\csname PY@tok@nv\endcsname{\def\PY@tc##1{\textcolor[rgb]{0.10,0.09,0.49}{##1}}}
\expandafter\def\csname PY@tok@no\endcsname{\def\PY@tc##1{\textcolor[rgb]{0.53,0.00,0.00}{##1}}}
\expandafter\def\csname PY@tok@nl\endcsname{\def\PY@tc##1{\textcolor[rgb]{0.63,0.63,0.00}{##1}}}
\expandafter\def\csname PY@tok@ni\endcsname{\let\PY@bf=\textbf\def\PY@tc##1{\textcolor[rgb]{0.60,0.60,0.60}{##1}}}
\expandafter\def\csname PY@tok@na\endcsname{\def\PY@tc##1{\textcolor[rgb]{0.49,0.56,0.16}{##1}}}
\expandafter\def\csname PY@tok@nt\endcsname{\let\PY@bf=\textbf\def\PY@tc##1{\textcolor[rgb]{0.00,0.50,0.00}{##1}}}
\expandafter\def\csname PY@tok@nd\endcsname{\def\PY@tc##1{\textcolor[rgb]{0.67,0.13,1.00}{##1}}}
\expandafter\def\csname PY@tok@s\endcsname{\def\PY@tc##1{\textcolor[rgb]{0.73,0.13,0.13}{##1}}}
\expandafter\def\csname PY@tok@sd\endcsname{\let\PY@it=\textit\def\PY@tc##1{\textcolor[rgb]{0.73,0.13,0.13}{##1}}}
\expandafter\def\csname PY@tok@si\endcsname{\let\PY@bf=\textbf\def\PY@tc##1{\textcolor[rgb]{0.73,0.40,0.53}{##1}}}
\expandafter\def\csname PY@tok@se\endcsname{\let\PY@bf=\textbf\def\PY@tc##1{\textcolor[rgb]{0.73,0.40,0.13}{##1}}}
\expandafter\def\csname PY@tok@sr\endcsname{\def\PY@tc##1{\textcolor[rgb]{0.73,0.40,0.53}{##1}}}
\expandafter\def\csname PY@tok@ss\endcsname{\def\PY@tc##1{\textcolor[rgb]{0.10,0.09,0.49}{##1}}}
\expandafter\def\csname PY@tok@sx\endcsname{\def\PY@tc##1{\textcolor[rgb]{0.00,0.50,0.00}{##1}}}
\expandafter\def\csname PY@tok@m\endcsname{\def\PY@tc##1{\textcolor[rgb]{0.40,0.40,0.40}{##1}}}
\expandafter\def\csname PY@tok@gh\endcsname{\let\PY@bf=\textbf\def\PY@tc##1{\textcolor[rgb]{0.00,0.00,0.50}{##1}}}
\expandafter\def\csname PY@tok@gu\endcsname{\let\PY@bf=\textbf\def\PY@tc##1{\textcolor[rgb]{0.50,0.00,0.50}{##1}}}
\expandafter\def\csname PY@tok@gd\endcsname{\def\PY@tc##1{\textcolor[rgb]{0.63,0.00,0.00}{##1}}}
\expandafter\def\csname PY@tok@gi\endcsname{\def\PY@tc##1{\textcolor[rgb]{0.00,0.63,0.00}{##1}}}
\expandafter\def\csname PY@tok@gr\endcsname{\def\PY@tc##1{\textcolor[rgb]{1.00,0.00,0.00}{##1}}}
\expandafter\def\csname PY@tok@ge\endcsname{\let\PY@it=\textit}
\expandafter\def\csname PY@tok@gs\endcsname{\let\PY@bf=\textbf}
\expandafter\def\csname PY@tok@gp\endcsname{\let\PY@bf=\textbf\def\PY@tc##1{\textcolor[rgb]{0.00,0.00,0.50}{##1}}}
\expandafter\def\csname PY@tok@go\endcsname{\def\PY@tc##1{\textcolor[rgb]{0.53,0.53,0.53}{##1}}}
\expandafter\def\csname PY@tok@gt\endcsname{\def\PY@tc##1{\textcolor[rgb]{0.00,0.27,0.87}{##1}}}
\expandafter\def\csname PY@tok@err\endcsname{\def\PY@bc##1{\setlength{\fboxsep}{0pt}\fcolorbox[rgb]{1.00,0.00,0.00}{1,1,1}{\strut ##1}}}
\expandafter\def\csname PY@tok@kc\endcsname{\let\PY@bf=\textbf\def\PY@tc##1{\textcolor[rgb]{0.00,0.50,0.00}{##1}}}
\expandafter\def\csname PY@tok@kd\endcsname{\let\PY@bf=\textbf\def\PY@tc##1{\textcolor[rgb]{0.00,0.50,0.00}{##1}}}
\expandafter\def\csname PY@tok@kn\endcsname{\let\PY@bf=\textbf\def\PY@tc##1{\textcolor[rgb]{0.00,0.50,0.00}{##1}}}
\expandafter\def\csname PY@tok@kr\endcsname{\let\PY@bf=\textbf\def\PY@tc##1{\textcolor[rgb]{0.00,0.50,0.00}{##1}}}
\expandafter\def\csname PY@tok@bp\endcsname{\def\PY@tc##1{\textcolor[rgb]{0.00,0.50,0.00}{##1}}}
\expandafter\def\csname PY@tok@fm\endcsname{\def\PY@tc##1{\textcolor[rgb]{0.00,0.00,1.00}{##1}}}
\expandafter\def\csname PY@tok@vc\endcsname{\def\PY@tc##1{\textcolor[rgb]{0.10,0.09,0.49}{##1}}}
\expandafter\def\csname PY@tok@vg\endcsname{\def\PY@tc##1{\textcolor[rgb]{0.10,0.09,0.49}{##1}}}
\expandafter\def\csname PY@tok@vi\endcsname{\def\PY@tc##1{\textcolor[rgb]{0.10,0.09,0.49}{##1}}}
\expandafter\def\csname PY@tok@vm\endcsname{\def\PY@tc##1{\textcolor[rgb]{0.10,0.09,0.49}{##1}}}
\expandafter\def\csname PY@tok@sa\endcsname{\def\PY@tc##1{\textcolor[rgb]{0.73,0.13,0.13}{##1}}}
\expandafter\def\csname PY@tok@sb\endcsname{\def\PY@tc##1{\textcolor[rgb]{0.73,0.13,0.13}{##1}}}
\expandafter\def\csname PY@tok@sc\endcsname{\def\PY@tc##1{\textcolor[rgb]{0.73,0.13,0.13}{##1}}}
\expandafter\def\csname PY@tok@dl\endcsname{\def\PY@tc##1{\textcolor[rgb]{0.73,0.13,0.13}{##1}}}
\expandafter\def\csname PY@tok@s2\endcsname{\def\PY@tc##1{\textcolor[rgb]{0.73,0.13,0.13}{##1}}}
\expandafter\def\csname PY@tok@sh\endcsname{\def\PY@tc##1{\textcolor[rgb]{0.73,0.13,0.13}{##1}}}
\expandafter\def\csname PY@tok@s1\endcsname{\def\PY@tc##1{\textcolor[rgb]{0.73,0.13,0.13}{##1}}}
\expandafter\def\csname PY@tok@mb\endcsname{\def\PY@tc##1{\textcolor[rgb]{0.40,0.40,0.40}{##1}}}
\expandafter\def\csname PY@tok@mf\endcsname{\def\PY@tc##1{\textcolor[rgb]{0.40,0.40,0.40}{##1}}}
\expandafter\def\csname PY@tok@mh\endcsname{\def\PY@tc##1{\textcolor[rgb]{0.40,0.40,0.40}{##1}}}
\expandafter\def\csname PY@tok@mi\endcsname{\def\PY@tc##1{\textcolor[rgb]{0.40,0.40,0.40}{##1}}}
\expandafter\def\csname PY@tok@il\endcsname{\def\PY@tc##1{\textcolor[rgb]{0.40,0.40,0.40}{##1}}}
\expandafter\def\csname PY@tok@mo\endcsname{\def\PY@tc##1{\textcolor[rgb]{0.40,0.40,0.40}{##1}}}
\expandafter\def\csname PY@tok@ch\endcsname{\let\PY@it=\textit\def\PY@tc##1{\textcolor[rgb]{0.25,0.50,0.50}{##1}}}
\expandafter\def\csname PY@tok@cm\endcsname{\let\PY@it=\textit\def\PY@tc##1{\textcolor[rgb]{0.25,0.50,0.50}{##1}}}
\expandafter\def\csname PY@tok@cpf\endcsname{\let\PY@it=\textit\def\PY@tc##1{\textcolor[rgb]{0.25,0.50,0.50}{##1}}}
\expandafter\def\csname PY@tok@c1\endcsname{\let\PY@it=\textit\def\PY@tc##1{\textcolor[rgb]{0.25,0.50,0.50}{##1}}}
\expandafter\def\csname PY@tok@cs\endcsname{\let\PY@it=\textit\def\PY@tc##1{\textcolor[rgb]{0.25,0.50,0.50}{##1}}}

\def\PYZbs{\char`\\}
\def\PYZus{\char`\_}
\def\PYZob{\char`\{}
\def\PYZcb{\char`\}}
\def\PYZca{\char`\^}
\def\PYZam{\char`\&}
\def\PYZlt{\char`\<}
\def\PYZgt{\char`\>}
\def\PYZsh{\char`\#}
\def\PYZpc{\char`\%}
\def\PYZdl{\char`\$}
\def\PYZhy{\char`\-}
\def\PYZsq{\char`\'}
\def\PYZdq{\char`\"}
\def\PYZti{\char`\~}
% for compatibility with earlier versions
\def\PYZat{@}
\def\PYZlb{[}
\def\PYZrb{]}
\makeatother


    % Exact colors from NB
    \definecolor{incolor}{rgb}{0.0, 0.0, 0.5}
    \definecolor{outcolor}{rgb}{0.545, 0.0, 0.0}



    
    % Prevent overflowing lines due to hard-to-break entities
    \sloppy 
    % Setup hyperref package
    \hypersetup{
      breaklinks=true,  % so long urls are correctly broken across lines
      colorlinks=true,
      urlcolor=urlcolor,
      linkcolor=linkcolor,
      citecolor=citecolor,
      }
    % Slightly bigger margins than the latex defaults
    
    \geometry{verbose,tmargin=1in,bmargin=1in,lmargin=1in,rmargin=1in}
    
    

    \begin{document}
    
    
    \maketitle
    
    

    
    \section{Foundations of Data Mining: Assignment
3}\label{foundations-of-data-mining-assignment-3}

Please complete all assignments in this notebook. You should submit this
notebook, as well as a PDF version (See File \textgreater{} Download
as).

\textbf{Deadline:} Thursday, March 29, 2018

    \begin{Verbatim}[commandchars=\\\{\}]
{\color{incolor}In [{\color{incolor}9}]:} \PY{o}{\PYZpc{}}\PY{k}{matplotlib} inline
        \PY{k+kn}{from} \PY{n+nn}{preamble} \PY{k}{import} \PY{o}{*}
        
        \PY{k+kn}{import} \PY{n+nn}{sklearn}\PY{n+nn}{.}\PY{n+nn}{decomposition} \PY{k}{as} \PY{n+nn}{deco}
        \PY{k+kn}{import} \PY{n+nn}{sklearn}\PY{n+nn}{.}\PY{n+nn}{manifold} \PY{k}{as} \PY{n+nn}{manifold}
        
        \PY{n}{plt}\PY{o}{.}\PY{n}{rcParams}\PY{p}{[}\PY{l+s+s1}{\PYZsq{}}\PY{l+s+s1}{savefig.dpi}\PY{l+s+s1}{\PYZsq{}}\PY{p}{]} \PY{o}{=} \PY{l+m+mi}{100} \PY{c+c1}{\PYZsh{} This controls the size of your figures}
        \PY{c+c1}{\PYZsh{} Comment out and restart notebook if you only want the last output of each cell.}
        \PY{n}{InteractiveShell}\PY{o}{.}\PY{n}{ast\PYZus{}node\PYZus{}interactivity} \PY{o}{=} \PY{l+s+s2}{\PYZdq{}}\PY{l+s+s2}{none}\PY{l+s+s2}{\PYZdq{}}\PY{c+c1}{\PYZsh{}\PYZdq{}all\PYZdq{} }
\end{Verbatim}


    \subsection{PCA and Isomap (5 Points,
1+2+2)}\label{pca-and-isomap-5-points-122}

Apply PCA and Isomap to images of handwritten digits (see below). You
may use sklearn.decomposition and sklearn.manifold.

\subsubsection{a)}\label{a}

Compute the first two components of the data using PCA. Make a scatter
plot of the data in the first two components of PCA indicating class
with color.

\subsubsection{b)}\label{b}

Compute an Isomap embedding with two components with nr\_neighbors=\{5,
50, N-1\} (three separate embeddings). For each of the Isomap
embeddings, apply the function "align" (see below) with "ref\_data" as
your computed pca embedding and "data" as the isomap embedding. Show a
scatter plot of each of the aligned isomap embeddings.

\subsubsection{c)}\label{c}

Visually compare how well the classes are separated in the different
scatter plots. What is the effect of changing the number of neighbors on
the score computed in the alignment function? What does it mean if the
score is zero? When do you expect the score to become zero and why?

More neighbours seems to decrease the computed score. As the number of
neighbours increases, the separation of the different classes seems to
become more vague. For n=5 the classes seem decently seperated, while
for n=50 and n=N-1 there is increasingly more overlap. If the score is
zero, it means that the matrix norm of the transformed dataset minus the
reference data (y) is zero for one of the 8 transformations. We expect
this to happen if the transformed dataset is equal to the reference data
for some transformation.

    \begin{Verbatim}[commandchars=\\\{\}]
{\color{incolor}In [{\color{incolor}10}]:} \PY{c+c1}{\PYZsh{} Load the data set}
         \PY{k+kn}{from} \PY{n+nn}{sklearn} \PY{k}{import} \PY{n}{datasets}
         \PY{n}{digits} \PY{o}{=} \PY{n}{datasets}\PY{o}{.}\PY{n}{load\PYZus{}digits}\PY{p}{(}\PY{n}{n\PYZus{}class}\PY{o}{=}\PY{l+m+mi}{10}\PY{p}{)}
         \PY{n}{X} \PY{o}{=} \PY{n}{digits}\PY{o}{.}\PY{n}{data}
         \PY{n}{y} \PY{o}{=} \PY{n}{digits}\PY{o}{.}\PY{n}{target}
         \PY{n}{N}\PY{o}{=}\PY{n+nb}{len}\PY{p}{(}\PY{n}{X}\PY{p}{)}
         
         \PY{c+c1}{\PYZsh{} Align a data set with a reference data set minimizing l\PYZus{}1 error}
         \PY{c+c1}{\PYZsh{} Returns aligned data set and alignment error}
         \PY{k}{def} \PY{n+nf}{align}\PY{p}{(}\PY{n}{ref\PYZus{}data}\PY{p}{,} \PY{n}{data}\PY{p}{)}\PY{p}{:}
             
             \PY{n}{transformations} \PY{o}{=} \PY{n}{np}\PY{o}{.}\PY{n}{asarray}\PY{p}{(}\PY{p}{[}
                 \PY{p}{[}\PY{p}{[}\PY{l+m+mi}{0}\PY{p}{,}\PY{l+m+mi}{1}\PY{p}{]}\PY{p}{,}\PY{p}{[}\PY{l+m+mi}{1}\PY{p}{,}\PY{l+m+mi}{0}\PY{p}{]}\PY{p}{]}\PY{p}{,} 
                 \PY{p}{[}\PY{p}{[}\PY{l+m+mi}{0}\PY{p}{,}\PY{o}{\PYZhy{}}\PY{l+m+mi}{1}\PY{p}{]}\PY{p}{,}\PY{p}{[}\PY{l+m+mi}{1}\PY{p}{,}\PY{l+m+mi}{0}\PY{p}{]}\PY{p}{]}\PY{p}{,} 
                 \PY{p}{[}\PY{p}{[}\PY{l+m+mi}{0}\PY{p}{,}\PY{l+m+mi}{1}\PY{p}{]}\PY{p}{,}\PY{p}{[}\PY{o}{\PYZhy{}}\PY{l+m+mi}{1}\PY{p}{,}\PY{l+m+mi}{0}\PY{p}{]}\PY{p}{]}\PY{p}{,} 
                 \PY{p}{[}\PY{p}{[}\PY{l+m+mi}{0}\PY{p}{,}\PY{o}{\PYZhy{}}\PY{l+m+mi}{1}\PY{p}{]}\PY{p}{,}\PY{p}{[}\PY{o}{\PYZhy{}}\PY{l+m+mi}{1}\PY{p}{,}\PY{l+m+mi}{0}\PY{p}{]}\PY{p}{]}\PY{p}{,} 
                 \PY{p}{[}\PY{p}{[}\PY{l+m+mi}{1}\PY{p}{,}\PY{l+m+mi}{0}\PY{p}{]}\PY{p}{,}\PY{p}{[}\PY{l+m+mi}{0}\PY{p}{,}\PY{l+m+mi}{1}\PY{p}{]}\PY{p}{]}\PY{p}{,} 
                 \PY{p}{[}\PY{p}{[}\PY{l+m+mi}{1}\PY{p}{,}\PY{l+m+mi}{0}\PY{p}{]}\PY{p}{,}\PY{p}{[}\PY{l+m+mi}{0}\PY{p}{,}\PY{o}{\PYZhy{}}\PY{l+m+mi}{1}\PY{p}{]}\PY{p}{]}\PY{p}{,} 
                 \PY{p}{[}\PY{p}{[}\PY{o}{\PYZhy{}}\PY{l+m+mi}{1}\PY{p}{,}\PY{l+m+mi}{0}\PY{p}{]}\PY{p}{,}\PY{p}{[}\PY{l+m+mi}{0}\PY{p}{,}\PY{l+m+mi}{1}\PY{p}{]}\PY{p}{]}\PY{p}{,} 
                 \PY{p}{[}\PY{p}{[}\PY{o}{\PYZhy{}}\PY{l+m+mi}{1}\PY{p}{,}\PY{l+m+mi}{0}\PY{p}{]}\PY{p}{,}\PY{p}{[}\PY{l+m+mi}{0}\PY{p}{,}\PY{o}{\PYZhy{}}\PY{l+m+mi}{1}\PY{p}{]}\PY{p}{]} 
             \PY{p}{]}\PY{p}{)}
             
             \PY{n}{score} \PY{o}{=} \PY{p}{[}\PY{p}{]}
             \PY{k}{for} \PY{n}{i} \PY{o+ow}{in} \PY{n+nb}{range}\PY{p}{(}\PY{l+m+mi}{0}\PY{p}{,}\PY{l+m+mi}{8}\PY{p}{)}\PY{p}{:}
                 \PY{n}{transf\PYZus{}data} \PY{o}{=}   \PY{n}{np}\PY{o}{.}\PY{n}{matmul}\PY{p}{(}\PY{n}{data}\PY{p}{,} \PY{n}{transformations}\PY{p}{[}\PY{n}{i}\PY{p}{]}\PY{p}{)}
                 \PY{n}{score}\PY{o}{.}\PY{n}{append}\PY{p}{(}\PY{n}{np}\PY{o}{.}\PY{n}{linalg}\PY{o}{.}\PY{n}{norm}\PY{p}{(} \PY{n}{transf\PYZus{}data} \PY{o}{\PYZhy{}} \PY{n}{ref\PYZus{}data}\PY{p}{,} \PY{n+nb}{ord}\PY{o}{=}\PY{l+m+mi}{1}\PY{p}{)} \PY{p}{)}
                 \PY{c+c1}{\PYZsh{}https://docs.scipy.org/doc/numpy\PYZhy{}1.13.0/reference/generated/numpy.linalg.norm.html}
                 
             \PY{n}{idx} \PY{o}{=} \PY{n}{np}\PY{o}{.}\PY{n}{argmin}\PY{p}{(}\PY{n}{score}\PY{p}{)}
             \PY{n}{transf\PYZus{}data} \PY{o}{=} \PY{n}{np}\PY{o}{.}\PY{n}{matmul}\PY{p}{(}\PY{n}{data}\PY{p}{,}\PY{n}{transformations}\PY{p}{[}\PY{n}{idx}\PY{p}{]}\PY{p}{)}
             
             \PY{n+nb}{print}\PY{p}{(}\PY{l+s+s2}{\PYZdq{}}\PY{l+s+s2}{Aligned the data sets. Score is }\PY{l+s+si}{\PYZob{}0:10.1f\PYZcb{}}\PY{l+s+s2}{  }\PY{l+s+s2}{\PYZdq{}}\PY{o}{.}\PY{n}{format}\PY{p}{(}\PY{n}{score}\PY{p}{[}\PY{n}{idx}\PY{p}{]}\PY{p}{)}\PY{p}{)}
             
             \PY{k}{return} \PY{n}{transf\PYZus{}data}\PY{p}{,} \PY{n}{score}\PY{p}{[}\PY{n}{idx}\PY{p}{]}
\end{Verbatim}


    \begin{Verbatim}[commandchars=\\\{\}]
{\color{incolor}In [{\color{incolor}11}]:} \PY{k+kn}{from} \PY{n+nn}{sklearn}\PY{n+nn}{.}\PY{n+nn}{decomposition} \PY{k}{import} \PY{n}{PCA}
         \PY{n}{pca} \PY{o}{=} \PY{n}{PCA}\PY{p}{(}\PY{n}{n\PYZus{}components}\PY{o}{=}\PY{l+m+mi}{2}\PY{p}{)}
         \PY{n}{X\PYZus{}r} \PY{o}{=} \PY{n}{pca}\PY{o}{.}\PY{n}{fit}\PY{p}{(}\PY{n}{X}\PY{p}{)}\PY{o}{.}\PY{n}{transform}\PY{p}{(}\PY{n}{X}\PY{p}{)}
         \PY{c+c1}{\PYZsh{}print(pca.explained\PYZus{}variance\PYZus{}ratio\PYZus{})  }
         \PY{c+c1}{\PYZsh{}print(pca.singular\PYZus{}values\PYZus{}) }
         
         \PY{c+c1}{\PYZsh{}print(X\PYZus{}r)}
         
         \PY{n+nb}{print}\PY{p}{(}\PY{l+s+s1}{\PYZsq{}}\PY{l+s+s1}{1 a}\PY{l+s+s1}{\PYZsq{}}\PY{p}{)}
         \PY{k}{def} \PY{n+nf}{printScatter1a}\PY{p}{(}\PY{p}{)}\PY{p}{:}
             \PY{n}{plt}\PY{o}{.}\PY{n}{figure}\PY{p}{(}\PY{p}{)}
         
             \PY{n}{colors} \PY{o}{=} \PY{p}{[}\PY{l+s+s1}{\PYZsq{}}\PY{l+s+s1}{navy}\PY{l+s+s1}{\PYZsq{}}\PY{p}{,} \PY{l+s+s1}{\PYZsq{}}\PY{l+s+s1}{turquoise}\PY{l+s+s1}{\PYZsq{}}\PY{p}{,}\PY{l+s+s1}{\PYZsq{}}\PY{l+s+s1}{darkorange}\PY{l+s+s1}{\PYZsq{}}\PY{p}{,} \PY{l+s+s1}{\PYZsq{}}\PY{l+s+s1}{blue}\PY{l+s+s1}{\PYZsq{}}\PY{p}{,} \PY{l+s+s1}{\PYZsq{}}\PY{l+s+s1}{red}\PY{l+s+s1}{\PYZsq{}}\PY{p}{,} \PY{l+s+s1}{\PYZsq{}}\PY{l+s+s1}{yellow}\PY{l+s+s1}{\PYZsq{}}\PY{p}{,} \PY{l+s+s1}{\PYZsq{}}\PY{l+s+s1}{xkcd:baby poop}\PY{l+s+s1}{\PYZsq{}}\PY{p}{,}
                       \PY{l+s+s1}{\PYZsq{}}\PY{l+s+s1}{xkcd:ugly blue}\PY{l+s+s1}{\PYZsq{}}\PY{p}{,} \PY{l+s+s1}{\PYZsq{}}\PY{l+s+s1}{xkcd:nasty green}\PY{l+s+s1}{\PYZsq{}}\PY{p}{,} \PY{l+s+s1}{\PYZsq{}}\PY{l+s+s1}{xkcd:battleship grey}\PY{l+s+s1}{\PYZsq{}}\PY{p}{]}\PY{p}{;}
             
             \PY{k}{for} \PY{n}{color}\PY{p}{,} \PY{n}{i} \PY{o+ow}{in} \PY{n+nb}{zip}\PY{p}{(}\PY{n}{colors}\PY{p}{,} \PY{p}{[}\PY{l+m+mi}{0}\PY{p}{,} \PY{l+m+mi}{1}\PY{p}{,} \PY{l+m+mi}{2}\PY{p}{,}\PY{l+m+mi}{3}\PY{p}{,}\PY{l+m+mi}{4}\PY{p}{,}\PY{l+m+mi}{5}\PY{p}{,}\PY{l+m+mi}{6}\PY{p}{,}\PY{l+m+mi}{7}\PY{p}{,}\PY{l+m+mi}{8}\PY{p}{,}\PY{l+m+mi}{9}\PY{p}{]}\PY{p}{)}\PY{p}{:}
                 \PY{n}{plt}\PY{o}{.}\PY{n}{scatter}\PY{p}{(}\PY{n}{X\PYZus{}r}\PY{p}{[}\PY{n}{y} \PY{o}{==} \PY{n}{i}\PY{p}{,} \PY{l+m+mi}{0}\PY{p}{]}\PY{p}{,} \PY{n}{X\PYZus{}r}\PY{p}{[}\PY{n}{y} \PY{o}{==} \PY{n}{i}\PY{p}{,} \PY{l+m+mi}{1}\PY{p}{]}\PY{p}{,} \PY{n}{color}\PY{o}{=}\PY{n}{color}\PY{p}{,} \PY{n}{alpha}\PY{o}{=}\PY{o}{.}\PY{l+m+mi}{8}\PY{p}{,} \PY{n}{lw}\PY{o}{=}\PY{l+m+mi}{2}\PY{p}{,}
                             \PY{n}{label}\PY{o}{=}\PY{n}{i}\PY{p}{)}\PY{p}{;}
             \PY{n}{plt}\PY{o}{.}\PY{n}{legend}\PY{p}{(}\PY{n}{loc}\PY{o}{=}\PY{l+s+s1}{\PYZsq{}}\PY{l+s+s1}{best}\PY{l+s+s1}{\PYZsq{}}\PY{p}{,} \PY{n}{shadow}\PY{o}{=}\PY{k+kc}{False}\PY{p}{,} \PY{n}{scatterpoints}\PY{o}{=}\PY{l+m+mi}{1}\PY{p}{)}\PY{p}{;}
         
             \PY{n}{plt}\PY{o}{.}\PY{n}{title}\PY{p}{(}\PY{l+s+s1}{\PYZsq{}}\PY{l+s+s1}{PCA of dataset}\PY{l+s+s1}{\PYZsq{}}\PY{p}{)}
             \PY{n}{plt}\PY{o}{.}\PY{n}{xlabel}\PY{p}{(}\PY{l+s+s1}{\PYZsq{}}\PY{l+s+s1}{Component 1}\PY{l+s+s1}{\PYZsq{}}\PY{p}{)}
             \PY{n}{plt}\PY{o}{.}\PY{n}{ylabel}\PY{p}{(}\PY{l+s+s1}{\PYZsq{}}\PY{l+s+s1}{Component 2}\PY{l+s+s1}{\PYZsq{}}\PY{p}{)}
             
         \PY{n}{printScatter1a}\PY{p}{(}\PY{p}{)}
\end{Verbatim}


    \begin{Verbatim}[commandchars=\\\{\}]
1 a

    \end{Verbatim}

    \begin{center}
    \adjustimage{max size={0.9\linewidth}{0.9\paperheight}}{output_4_1.pdf}
    \end{center}
    { \hspace*{\fill} \\}
    
    \begin{Verbatim}[commandchars=\\\{\}]
{\color{incolor}In [{\color{incolor}13}]:} \PY{c+c1}{\PYZsh{} isomap embedding}
         
         \PY{k+kn}{from} \PY{n+nn}{matplotlib} \PY{k}{import} \PY{n}{offsetbox}
         \PY{k+kn}{from} \PY{n+nn}{matplotlib}\PY{n+nn}{.}\PY{n+nn}{colors} \PY{k}{import} \PY{n}{Colormap}
         \PY{k+kn}{from} \PY{n+nn}{sklearn} \PY{k}{import} \PY{p}{(}\PY{n}{manifold}\PY{p}{,} \PY{n}{datasets}\PY{p}{,} \PY{n}{decomposition}\PY{p}{,} \PY{n}{ensemble}\PY{p}{,}
                              \PY{n}{discriminant\PYZus{}analysis}\PY{p}{,} \PY{n}{random\PYZus{}projection}\PY{p}{)}
         
         \PY{n}{title} \PY{o}{=} \PY{p}{[}\PY{l+s+s1}{\PYZsq{}}\PY{l+s+s1}{n\PYZus{}neighbours = 5}\PY{l+s+s1}{\PYZsq{}}\PY{p}{,} \PY{l+s+s1}{\PYZsq{}}\PY{l+s+s1}{n\PYZus{}neighbours = 50}\PY{l+s+s1}{\PYZsq{}}\PY{p}{,} \PY{l+s+s1}{\PYZsq{}}\PY{l+s+s1}{n\PYZus{}neighbours = N\PYZhy{}1}\PY{l+s+s1}{\PYZsq{}}\PY{p}{]}\PY{p}{;}    
         \PY{n}{values} \PY{o}{=} \PY{p}{[}\PY{l+m+mi}{5}\PY{p}{,}\PY{l+m+mi}{50}\PY{p}{,}\PY{n}{N}\PY{o}{\PYZhy{}}\PY{l+m+mi}{1}\PY{p}{]}\PY{p}{;}
         
         \PY{n+nb}{print}\PY{p}{(}\PY{l+s+s1}{\PYZsq{}}\PY{l+s+s1}{1 b}\PY{l+s+s1}{\PYZsq{}}\PY{p}{)}   
         \PY{k}{for} \PY{n}{n}\PY{p}{,} \PY{n}{title}  \PY{o+ow}{in} \PY{n+nb}{zip}\PY{p}{(}\PY{n}{values}\PY{p}{,}\PY{n}{title}\PY{p}{)} \PY{p}{:}
             \PY{n+nb}{print}\PY{p}{(}\PY{n}{title}\PY{p}{)}\PY{p}{;}
             \PY{c+c1}{\PYZsh{}print(n);}
             \PY{n}{X\PYZus{}iso} \PY{o}{=} \PY{n}{manifold}\PY{o}{.}\PY{n}{Isomap}\PY{p}{(}\PY{n}{n\PYZus{}neighbors}\PY{o}{=}\PY{n}{n}\PY{p}{,} \PY{n}{n\PYZus{}components}\PY{o}{=}\PY{l+m+mi}{2}\PY{p}{)}\PY{o}{.}\PY{n}{fit\PYZus{}transform}\PY{p}{(}\PY{n}{X}\PY{p}{)}
         
             \PY{n}{X\PYZus{}aligned}\PY{p}{,}\PY{n}{score} \PY{o}{=} \PY{n}{align}\PY{p}{(}\PY{n}{X\PYZus{}r}\PY{p}{,}\PY{n}{X\PYZus{}iso}\PY{p}{)}
         
             \PY{n}{plt}\PY{o}{.}\PY{n}{figure}\PY{p}{(}\PY{p}{)}\PY{p}{;}
             \PY{n}{plt}\PY{o}{.}\PY{n}{title}\PY{p}{(}\PY{n}{title}\PY{p}{)}\PY{p}{;}
             \PY{c+c1}{\PYZsh{}plt.scatter(X\PYZus{}iso[:,0],X\PYZus{}iso[:,1])}
             \PY{c+c1}{\PYZsh{}plt.scatter(X\PYZus{}aligned[:,0],X\PYZus{}aligned[:,1])}
             
             \PY{n}{colors} \PY{o}{=} \PY{p}{[}\PY{l+s+s1}{\PYZsq{}}\PY{l+s+s1}{navy}\PY{l+s+s1}{\PYZsq{}}\PY{p}{,} \PY{l+s+s1}{\PYZsq{}}\PY{l+s+s1}{turquoise}\PY{l+s+s1}{\PYZsq{}}\PY{p}{,}\PY{l+s+s1}{\PYZsq{}}\PY{l+s+s1}{darkorange}\PY{l+s+s1}{\PYZsq{}}\PY{p}{,} \PY{l+s+s1}{\PYZsq{}}\PY{l+s+s1}{blue}\PY{l+s+s1}{\PYZsq{}}\PY{p}{,} \PY{l+s+s1}{\PYZsq{}}\PY{l+s+s1}{red}\PY{l+s+s1}{\PYZsq{}}\PY{p}{,} \PY{l+s+s1}{\PYZsq{}}\PY{l+s+s1}{yellow}\PY{l+s+s1}{\PYZsq{}}\PY{p}{,} \PY{l+s+s1}{\PYZsq{}}\PY{l+s+s1}{xkcd:baby poop}\PY{l+s+s1}{\PYZsq{}}\PY{p}{,}
                       \PY{l+s+s1}{\PYZsq{}}\PY{l+s+s1}{xkcd:ugly blue}\PY{l+s+s1}{\PYZsq{}}\PY{p}{,} \PY{l+s+s1}{\PYZsq{}}\PY{l+s+s1}{xkcd:nasty green}\PY{l+s+s1}{\PYZsq{}}\PY{p}{,} \PY{l+s+s1}{\PYZsq{}}\PY{l+s+s1}{xkcd:battleship grey}\PY{l+s+s1}{\PYZsq{}}\PY{p}{]}\PY{p}{;}
             
             \PY{k}{for} \PY{n}{color}\PY{p}{,} \PY{n}{i} \PY{o+ow}{in} \PY{n+nb}{zip}\PY{p}{(}\PY{n}{colors}\PY{p}{,} \PY{p}{[}\PY{l+m+mi}{0}\PY{p}{,} \PY{l+m+mi}{1}\PY{p}{,} \PY{l+m+mi}{2}\PY{p}{,}\PY{l+m+mi}{3}\PY{p}{,}\PY{l+m+mi}{4}\PY{p}{,}\PY{l+m+mi}{5}\PY{p}{,}\PY{l+m+mi}{6}\PY{p}{,}\PY{l+m+mi}{7}\PY{p}{,}\PY{l+m+mi}{8}\PY{p}{,}\PY{l+m+mi}{9}\PY{p}{]}\PY{p}{)}\PY{p}{:}
                 \PY{n}{plt}\PY{o}{.}\PY{n}{scatter}\PY{p}{(}\PY{n}{X\PYZus{}aligned}\PY{p}{[}\PY{n}{y} \PY{o}{==} \PY{n}{i}\PY{p}{,} \PY{l+m+mi}{0}\PY{p}{]}\PY{p}{,} \PY{n}{X\PYZus{}aligned}\PY{p}{[}\PY{n}{y} \PY{o}{==} \PY{n}{i}\PY{p}{,} \PY{l+m+mi}{1}\PY{p}{]}\PY{p}{,} \PY{n}{color}\PY{o}{=}\PY{n}{color}\PY{p}{,} \PY{n}{alpha}\PY{o}{=}\PY{o}{.}\PY{l+m+mi}{8}\PY{p}{,} \PY{n}{lw}\PY{o}{=}\PY{l+m+mi}{2}\PY{p}{,}
                             \PY{n}{label}\PY{o}{=}\PY{n}{i}\PY{p}{)}\PY{p}{;}
             \PY{n}{plt}\PY{o}{.}\PY{n}{legend}\PY{p}{(}\PY{n}{loc}\PY{o}{=}\PY{l+s+s1}{\PYZsq{}}\PY{l+s+s1}{best}\PY{l+s+s1}{\PYZsq{}}\PY{p}{,} \PY{n}{shadow}\PY{o}{=}\PY{k+kc}{False}\PY{p}{,} \PY{n}{scatterpoints}\PY{o}{=}\PY{l+m+mi}{1}\PY{p}{)}\PY{p}{;}
             \PY{n}{plt}\PY{o}{.}\PY{n}{xlabel}\PY{p}{(}\PY{l+s+s1}{\PYZsq{}}\PY{l+s+s1}{Component 1}\PY{l+s+s1}{\PYZsq{}}\PY{p}{)}
             \PY{n}{plt}\PY{o}{.}\PY{n}{ylabel}\PY{p}{(}\PY{l+s+s1}{\PYZsq{}}\PY{l+s+s1}{Component 2}\PY{l+s+s1}{\PYZsq{}}\PY{p}{)}
             \PY{n}{plt}\PY{o}{.}\PY{n}{title}\PY{p}{(}\PY{n}{title}\PY{p}{)}\PY{p}{;}
\end{Verbatim}


    \begin{Verbatim}[commandchars=\\\{\}]
1 b
n\_neighbours = 5
Aligned the data sets. Score is   102234.2  
n\_neighbours = 50
Aligned the data sets. Score is    31067.5  
n\_neighbours = N-1
Aligned the data sets. Score is        0.0  

    \end{Verbatim}

    \begin{center}
    \adjustimage{max size={0.9\linewidth}{0.9\paperheight}}{output_5_1.pdf}
    \end{center}
    { \hspace*{\fill} \\}
    
    \begin{center}
    \adjustimage{max size={0.9\linewidth}{0.9\paperheight}}{output_5_2.pdf}
    \end{center}
    { \hspace*{\fill} \\}
    
    \begin{center}
    \adjustimage{max size={0.9\linewidth}{0.9\paperheight}}{output_5_3.pdf}
    \end{center}
    { \hspace*{\fill} \\}
    
    \subsection{Classical Multidimensional Scaling (6 Points,
1+2+2+1)}\label{classical-multidimensional-scaling-6-points-1221}

Show that for mean-centered data sets we can recover inner products
using pairwise distance information only. This is used by the isomap
embedding algorithm.

We are given all squared pairwise distances of an otherwise unknown
point set \({\bf p}_1,\dots, {\bf p}_n \in \mathbb{R}^d\), i.e., we are
given for all \(1 \leq i,j \leq n\) the values

\[  d_{ij} = \|{\bf p}_i - {\bf p}_j\|^2. \]

We assume that the point set is mean-centered, that is

\[ \sum_{i=1}^{n} {\bf p}_i = \vec{{\bf 0}}.\]

(where \(\vec{{\bf 0}}\) is the vector of zeros)

In the following, \(\langle {\bf p}_i , {\bf p}_j \rangle\) denotes the
inner product of \({\bf p}_i\) and \({\bf p}_j\). Prove that the
following holds true for mean-centered point sets:

\[-2 \langle {\bf p}_i , {\bf p}_j \rangle = 
d_{ij}
- \sum_{\ell=1}^{n} \frac{d_{i\ell}}{n}   
-  \sum_{\ell=1}^{n} \frac{d_{j\ell}}{n} 
+ \sum_{k=1}^{n}~ \sum_{\ell=1}^{n} \frac{d_{k\ell}}{n^2} 
\]

You may use the following steps in your derivation.

\subsubsection{a)}\label{a}

Expand \(d_{ij}\) to yield an expression of
\(\langle {\bf p}_i, {\bf p}_j \rangle\), \(\|{\bf p}_i\|^2\) and
\(\|{\bf p}_j\|^2\).

\subsubsection{b)}\label{b}

Show that the following holds for any \$ \{\bf q\}
\in \mathbb{R}\^{}d\$:
\[ \sum_{1 \leq i \leq n} \langle {\bf p}_i , {\bf q} \rangle =   0\]

\subsubsection{c)}\label{c}

Prove that
\[ \|{\bf p}_i\|^2 = \sum_{\ell=1}^{n} \frac{d_{i\ell}}{n} - \sum_{k=1}^{n}~ \sum_{\ell=1}^{n} \frac{d_{k\ell}}{2n^2}\]

\subsubsection{d)}\label{d}

Combine the steps in your proof.

    \subsection{Locality-sensitive hashing (4 Points,
2+1+1)}\label{locality-sensitive-hashing-4-points-211}

\subsubsection{a)}\label{a}

Prove that if the Jaccard Similarity of two sets is \(0\), then
minhashing always gives a correct estimate of the Jaccard similarity

\subsubsection{b)}\label{b}

Let \(H\) be a family of \((d_1,d_2,p_1,p_2)\)-locality-sensitive hash
functions. Assume that \(p_2=0\) and assume we have a total numer of
\(m\) hash functions from this family available. Which combination of
AND-constructions and OR-constructions should we use to maximally
amplify the hash family?

\subsubsection{(c)}\label{c}

Let \(H\) be a family of \((d_1,d_2,p_1,p_2)\)-locality-sensitive hash
functions. Assume that \(p_2=\frac{1}{n}\) and assume we have \(n\) data
points \(\bf P\) which are stored in a hash table using a randomly
chosen function \(h\) from \(H\). Given a query point \(\bf q\), we
retrieve the points in the hash bucket with index \(h(\bf q)\) to search
for a point which has small distance to \(\bf q\). Let \(X\) be a random
variable that is equal to the size of the set

\[
\begin{aligned}
\left\{{\bf p \in P}~:~ h({\bf p})=h({\bf q}) ~\wedge~ d({\bf p,q}) \geq d_2\right\}\
\end{aligned}
\]

which consists of the false positives of this query. Derive the expected
number of false-positives \(E\left[ X \right]\).


    % Add a bibliography block to the postdoc
    
    
    
    \end{document}
