
% Default to the notebook output style

    


% Inherit from the specified cell style.




    
\documentclass[11pt]{article}

    
    
    \usepackage[T1]{fontenc}
    % Nicer default font (+ math font) than Computer Modern for most use cases
    \usepackage{mathpazo}

    % Basic figure setup, for now with no caption control since it's done
    % automatically by Pandoc (which extracts ![](path) syntax from Markdown).
    \usepackage{graphicx}
    % We will generate all images so they have a width \maxwidth. This means
    % that they will get their normal width if they fit onto the page, but
    % are scaled down if they would overflow the margins.
    \makeatletter
    \def\maxwidth{\ifdim\Gin@nat@width>\linewidth\linewidth
    \else\Gin@nat@width\fi}
    \makeatother
    \let\Oldincludegraphics\includegraphics
    % Set max figure width to be 80% of text width, for now hardcoded.
    \renewcommand{\includegraphics}[1]{\Oldincludegraphics[width=.8\maxwidth]{#1}}
    % Ensure that by default, figures have no caption (until we provide a
    % proper Figure object with a Caption API and a way to capture that
    % in the conversion process - todo).
    \usepackage{caption}
    \DeclareCaptionLabelFormat{nolabel}{}
    \captionsetup{labelformat=nolabel}

    \usepackage{adjustbox} % Used to constrain images to a maximum size 
    \usepackage{xcolor} % Allow colors to be defined
    \usepackage{enumerate} % Needed for markdown enumerations to work
    \usepackage{geometry} % Used to adjust the document margins
    \usepackage{amsmath} % Equations
    \usepackage{amssymb} % Equations
    \usepackage{textcomp} % defines textquotesingle
    % Hack from http://tex.stackexchange.com/a/47451/13684:
    \AtBeginDocument{%
        \def\PYZsq{\textquotesingle}% Upright quotes in Pygmentized code
    }
    \usepackage{upquote} % Upright quotes for verbatim code
    \usepackage{eurosym} % defines \euro
    \usepackage[mathletters]{ucs} % Extended unicode (utf-8) support
    \usepackage[utf8x]{inputenc} % Allow utf-8 characters in the tex document
    \usepackage{fancyvrb} % verbatim replacement that allows latex
    \usepackage{grffile} % extends the file name processing of package graphics 
                         % to support a larger range 
    % The hyperref package gives us a pdf with properly built
    % internal navigation ('pdf bookmarks' for the table of contents,
    % internal cross-reference links, web links for URLs, etc.)
    \usepackage{hyperref}
    \usepackage{longtable} % longtable support required by pandoc >1.10
    \usepackage{booktabs}  % table support for pandoc > 1.12.2
    \usepackage[inline]{enumitem} % IRkernel/repr support (it uses the enumerate* environment)
    \usepackage[normalem]{ulem} % ulem is needed to support strikethroughs (\sout)
                                % normalem makes italics be italics, not underlines
    

    
    
    % Colors for the hyperref package
    \definecolor{urlcolor}{rgb}{0,.145,.698}
    \definecolor{linkcolor}{rgb}{.71,0.21,0.01}
    \definecolor{citecolor}{rgb}{.12,.54,.11}

    % ANSI colors
    \definecolor{ansi-black}{HTML}{3E424D}
    \definecolor{ansi-black-intense}{HTML}{282C36}
    \definecolor{ansi-red}{HTML}{E75C58}
    \definecolor{ansi-red-intense}{HTML}{B22B31}
    \definecolor{ansi-green}{HTML}{00A250}
    \definecolor{ansi-green-intense}{HTML}{007427}
    \definecolor{ansi-yellow}{HTML}{DDB62B}
    \definecolor{ansi-yellow-intense}{HTML}{B27D12}
    \definecolor{ansi-blue}{HTML}{208FFB}
    \definecolor{ansi-blue-intense}{HTML}{0065CA}
    \definecolor{ansi-magenta}{HTML}{D160C4}
    \definecolor{ansi-magenta-intense}{HTML}{A03196}
    \definecolor{ansi-cyan}{HTML}{60C6C8}
    \definecolor{ansi-cyan-intense}{HTML}{258F8F}
    \definecolor{ansi-white}{HTML}{C5C1B4}
    \definecolor{ansi-white-intense}{HTML}{A1A6B2}

    % commands and environments needed by pandoc snippets
    % extracted from the output of `pandoc -s`
    \providecommand{\tightlist}{%
      \setlength{\itemsep}{0pt}\setlength{\parskip}{0pt}}
    \DefineVerbatimEnvironment{Highlighting}{Verbatim}{commandchars=\\\{\}}
    % Add ',fontsize=\small' for more characters per line
    \newenvironment{Shaded}{}{}
    \newcommand{\KeywordTok}[1]{\textcolor[rgb]{0.00,0.44,0.13}{\textbf{{#1}}}}
    \newcommand{\DataTypeTok}[1]{\textcolor[rgb]{0.56,0.13,0.00}{{#1}}}
    \newcommand{\DecValTok}[1]{\textcolor[rgb]{0.25,0.63,0.44}{{#1}}}
    \newcommand{\BaseNTok}[1]{\textcolor[rgb]{0.25,0.63,0.44}{{#1}}}
    \newcommand{\FloatTok}[1]{\textcolor[rgb]{0.25,0.63,0.44}{{#1}}}
    \newcommand{\CharTok}[1]{\textcolor[rgb]{0.25,0.44,0.63}{{#1}}}
    \newcommand{\StringTok}[1]{\textcolor[rgb]{0.25,0.44,0.63}{{#1}}}
    \newcommand{\CommentTok}[1]{\textcolor[rgb]{0.38,0.63,0.69}{\textit{{#1}}}}
    \newcommand{\OtherTok}[1]{\textcolor[rgb]{0.00,0.44,0.13}{{#1}}}
    \newcommand{\AlertTok}[1]{\textcolor[rgb]{1.00,0.00,0.00}{\textbf{{#1}}}}
    \newcommand{\FunctionTok}[1]{\textcolor[rgb]{0.02,0.16,0.49}{{#1}}}
    \newcommand{\RegionMarkerTok}[1]{{#1}}
    \newcommand{\ErrorTok}[1]{\textcolor[rgb]{1.00,0.00,0.00}{\textbf{{#1}}}}
    \newcommand{\NormalTok}[1]{{#1}}
    
    % Additional commands for more recent versions of Pandoc
    \newcommand{\ConstantTok}[1]{\textcolor[rgb]{0.53,0.00,0.00}{{#1}}}
    \newcommand{\SpecialCharTok}[1]{\textcolor[rgb]{0.25,0.44,0.63}{{#1}}}
    \newcommand{\VerbatimStringTok}[1]{\textcolor[rgb]{0.25,0.44,0.63}{{#1}}}
    \newcommand{\SpecialStringTok}[1]{\textcolor[rgb]{0.73,0.40,0.53}{{#1}}}
    \newcommand{\ImportTok}[1]{{#1}}
    \newcommand{\DocumentationTok}[1]{\textcolor[rgb]{0.73,0.13,0.13}{\textit{{#1}}}}
    \newcommand{\AnnotationTok}[1]{\textcolor[rgb]{0.38,0.63,0.69}{\textbf{\textit{{#1}}}}}
    \newcommand{\CommentVarTok}[1]{\textcolor[rgb]{0.38,0.63,0.69}{\textbf{\textit{{#1}}}}}
    \newcommand{\VariableTok}[1]{\textcolor[rgb]{0.10,0.09,0.49}{{#1}}}
    \newcommand{\ControlFlowTok}[1]{\textcolor[rgb]{0.00,0.44,0.13}{\textbf{{#1}}}}
    \newcommand{\OperatorTok}[1]{\textcolor[rgb]{0.40,0.40,0.40}{{#1}}}
    \newcommand{\BuiltInTok}[1]{{#1}}
    \newcommand{\ExtensionTok}[1]{{#1}}
    \newcommand{\PreprocessorTok}[1]{\textcolor[rgb]{0.74,0.48,0.00}{{#1}}}
    \newcommand{\AttributeTok}[1]{\textcolor[rgb]{0.49,0.56,0.16}{{#1}}}
    \newcommand{\InformationTok}[1]{\textcolor[rgb]{0.38,0.63,0.69}{\textbf{\textit{{#1}}}}}
    \newcommand{\WarningTok}[1]{\textcolor[rgb]{0.38,0.63,0.69}{\textbf{\textit{{#1}}}}}
    
    
    % Define a nice break command that doesn't care if a line doesn't already
    % exist.
    \def\br{\hspace*{\fill} \\* }
    % Math Jax compatability definitions
    \def\gt{>}
    \def\lt{<}
    % Document parameters
    \title{Assignment 1}
    
    
    

    % Pygments definitions
    
\makeatletter
\def\PY@reset{\let\PY@it=\relax \let\PY@bf=\relax%
    \let\PY@ul=\relax \let\PY@tc=\relax%
    \let\PY@bc=\relax \let\PY@ff=\relax}
\def\PY@tok#1{\csname PY@tok@#1\endcsname}
\def\PY@toks#1+{\ifx\relax#1\empty\else%
    \PY@tok{#1}\expandafter\PY@toks\fi}
\def\PY@do#1{\PY@bc{\PY@tc{\PY@ul{%
    \PY@it{\PY@bf{\PY@ff{#1}}}}}}}
\def\PY#1#2{\PY@reset\PY@toks#1+\relax+\PY@do{#2}}

\expandafter\def\csname PY@tok@w\endcsname{\def\PY@tc##1{\textcolor[rgb]{0.73,0.73,0.73}{##1}}}
\expandafter\def\csname PY@tok@c\endcsname{\let\PY@it=\textit\def\PY@tc##1{\textcolor[rgb]{0.25,0.50,0.50}{##1}}}
\expandafter\def\csname PY@tok@cp\endcsname{\def\PY@tc##1{\textcolor[rgb]{0.74,0.48,0.00}{##1}}}
\expandafter\def\csname PY@tok@k\endcsname{\let\PY@bf=\textbf\def\PY@tc##1{\textcolor[rgb]{0.00,0.50,0.00}{##1}}}
\expandafter\def\csname PY@tok@kp\endcsname{\def\PY@tc##1{\textcolor[rgb]{0.00,0.50,0.00}{##1}}}
\expandafter\def\csname PY@tok@kt\endcsname{\def\PY@tc##1{\textcolor[rgb]{0.69,0.00,0.25}{##1}}}
\expandafter\def\csname PY@tok@o\endcsname{\def\PY@tc##1{\textcolor[rgb]{0.40,0.40,0.40}{##1}}}
\expandafter\def\csname PY@tok@ow\endcsname{\let\PY@bf=\textbf\def\PY@tc##1{\textcolor[rgb]{0.67,0.13,1.00}{##1}}}
\expandafter\def\csname PY@tok@nb\endcsname{\def\PY@tc##1{\textcolor[rgb]{0.00,0.50,0.00}{##1}}}
\expandafter\def\csname PY@tok@nf\endcsname{\def\PY@tc##1{\textcolor[rgb]{0.00,0.00,1.00}{##1}}}
\expandafter\def\csname PY@tok@nc\endcsname{\let\PY@bf=\textbf\def\PY@tc##1{\textcolor[rgb]{0.00,0.00,1.00}{##1}}}
\expandafter\def\csname PY@tok@nn\endcsname{\let\PY@bf=\textbf\def\PY@tc##1{\textcolor[rgb]{0.00,0.00,1.00}{##1}}}
\expandafter\def\csname PY@tok@ne\endcsname{\let\PY@bf=\textbf\def\PY@tc##1{\textcolor[rgb]{0.82,0.25,0.23}{##1}}}
\expandafter\def\csname PY@tok@nv\endcsname{\def\PY@tc##1{\textcolor[rgb]{0.10,0.09,0.49}{##1}}}
\expandafter\def\csname PY@tok@no\endcsname{\def\PY@tc##1{\textcolor[rgb]{0.53,0.00,0.00}{##1}}}
\expandafter\def\csname PY@tok@nl\endcsname{\def\PY@tc##1{\textcolor[rgb]{0.63,0.63,0.00}{##1}}}
\expandafter\def\csname PY@tok@ni\endcsname{\let\PY@bf=\textbf\def\PY@tc##1{\textcolor[rgb]{0.60,0.60,0.60}{##1}}}
\expandafter\def\csname PY@tok@na\endcsname{\def\PY@tc##1{\textcolor[rgb]{0.49,0.56,0.16}{##1}}}
\expandafter\def\csname PY@tok@nt\endcsname{\let\PY@bf=\textbf\def\PY@tc##1{\textcolor[rgb]{0.00,0.50,0.00}{##1}}}
\expandafter\def\csname PY@tok@nd\endcsname{\def\PY@tc##1{\textcolor[rgb]{0.67,0.13,1.00}{##1}}}
\expandafter\def\csname PY@tok@s\endcsname{\def\PY@tc##1{\textcolor[rgb]{0.73,0.13,0.13}{##1}}}
\expandafter\def\csname PY@tok@sd\endcsname{\let\PY@it=\textit\def\PY@tc##1{\textcolor[rgb]{0.73,0.13,0.13}{##1}}}
\expandafter\def\csname PY@tok@si\endcsname{\let\PY@bf=\textbf\def\PY@tc##1{\textcolor[rgb]{0.73,0.40,0.53}{##1}}}
\expandafter\def\csname PY@tok@se\endcsname{\let\PY@bf=\textbf\def\PY@tc##1{\textcolor[rgb]{0.73,0.40,0.13}{##1}}}
\expandafter\def\csname PY@tok@sr\endcsname{\def\PY@tc##1{\textcolor[rgb]{0.73,0.40,0.53}{##1}}}
\expandafter\def\csname PY@tok@ss\endcsname{\def\PY@tc##1{\textcolor[rgb]{0.10,0.09,0.49}{##1}}}
\expandafter\def\csname PY@tok@sx\endcsname{\def\PY@tc##1{\textcolor[rgb]{0.00,0.50,0.00}{##1}}}
\expandafter\def\csname PY@tok@m\endcsname{\def\PY@tc##1{\textcolor[rgb]{0.40,0.40,0.40}{##1}}}
\expandafter\def\csname PY@tok@gh\endcsname{\let\PY@bf=\textbf\def\PY@tc##1{\textcolor[rgb]{0.00,0.00,0.50}{##1}}}
\expandafter\def\csname PY@tok@gu\endcsname{\let\PY@bf=\textbf\def\PY@tc##1{\textcolor[rgb]{0.50,0.00,0.50}{##1}}}
\expandafter\def\csname PY@tok@gd\endcsname{\def\PY@tc##1{\textcolor[rgb]{0.63,0.00,0.00}{##1}}}
\expandafter\def\csname PY@tok@gi\endcsname{\def\PY@tc##1{\textcolor[rgb]{0.00,0.63,0.00}{##1}}}
\expandafter\def\csname PY@tok@gr\endcsname{\def\PY@tc##1{\textcolor[rgb]{1.00,0.00,0.00}{##1}}}
\expandafter\def\csname PY@tok@ge\endcsname{\let\PY@it=\textit}
\expandafter\def\csname PY@tok@gs\endcsname{\let\PY@bf=\textbf}
\expandafter\def\csname PY@tok@gp\endcsname{\let\PY@bf=\textbf\def\PY@tc##1{\textcolor[rgb]{0.00,0.00,0.50}{##1}}}
\expandafter\def\csname PY@tok@go\endcsname{\def\PY@tc##1{\textcolor[rgb]{0.53,0.53,0.53}{##1}}}
\expandafter\def\csname PY@tok@gt\endcsname{\def\PY@tc##1{\textcolor[rgb]{0.00,0.27,0.87}{##1}}}
\expandafter\def\csname PY@tok@err\endcsname{\def\PY@bc##1{\setlength{\fboxsep}{0pt}\fcolorbox[rgb]{1.00,0.00,0.00}{1,1,1}{\strut ##1}}}
\expandafter\def\csname PY@tok@kc\endcsname{\let\PY@bf=\textbf\def\PY@tc##1{\textcolor[rgb]{0.00,0.50,0.00}{##1}}}
\expandafter\def\csname PY@tok@kd\endcsname{\let\PY@bf=\textbf\def\PY@tc##1{\textcolor[rgb]{0.00,0.50,0.00}{##1}}}
\expandafter\def\csname PY@tok@kn\endcsname{\let\PY@bf=\textbf\def\PY@tc##1{\textcolor[rgb]{0.00,0.50,0.00}{##1}}}
\expandafter\def\csname PY@tok@kr\endcsname{\let\PY@bf=\textbf\def\PY@tc##1{\textcolor[rgb]{0.00,0.50,0.00}{##1}}}
\expandafter\def\csname PY@tok@bp\endcsname{\def\PY@tc##1{\textcolor[rgb]{0.00,0.50,0.00}{##1}}}
\expandafter\def\csname PY@tok@fm\endcsname{\def\PY@tc##1{\textcolor[rgb]{0.00,0.00,1.00}{##1}}}
\expandafter\def\csname PY@tok@vc\endcsname{\def\PY@tc##1{\textcolor[rgb]{0.10,0.09,0.49}{##1}}}
\expandafter\def\csname PY@tok@vg\endcsname{\def\PY@tc##1{\textcolor[rgb]{0.10,0.09,0.49}{##1}}}
\expandafter\def\csname PY@tok@vi\endcsname{\def\PY@tc##1{\textcolor[rgb]{0.10,0.09,0.49}{##1}}}
\expandafter\def\csname PY@tok@vm\endcsname{\def\PY@tc##1{\textcolor[rgb]{0.10,0.09,0.49}{##1}}}
\expandafter\def\csname PY@tok@sa\endcsname{\def\PY@tc##1{\textcolor[rgb]{0.73,0.13,0.13}{##1}}}
\expandafter\def\csname PY@tok@sb\endcsname{\def\PY@tc##1{\textcolor[rgb]{0.73,0.13,0.13}{##1}}}
\expandafter\def\csname PY@tok@sc\endcsname{\def\PY@tc##1{\textcolor[rgb]{0.73,0.13,0.13}{##1}}}
\expandafter\def\csname PY@tok@dl\endcsname{\def\PY@tc##1{\textcolor[rgb]{0.73,0.13,0.13}{##1}}}
\expandafter\def\csname PY@tok@s2\endcsname{\def\PY@tc##1{\textcolor[rgb]{0.73,0.13,0.13}{##1}}}
\expandafter\def\csname PY@tok@sh\endcsname{\def\PY@tc##1{\textcolor[rgb]{0.73,0.13,0.13}{##1}}}
\expandafter\def\csname PY@tok@s1\endcsname{\def\PY@tc##1{\textcolor[rgb]{0.73,0.13,0.13}{##1}}}
\expandafter\def\csname PY@tok@mb\endcsname{\def\PY@tc##1{\textcolor[rgb]{0.40,0.40,0.40}{##1}}}
\expandafter\def\csname PY@tok@mf\endcsname{\def\PY@tc##1{\textcolor[rgb]{0.40,0.40,0.40}{##1}}}
\expandafter\def\csname PY@tok@mh\endcsname{\def\PY@tc##1{\textcolor[rgb]{0.40,0.40,0.40}{##1}}}
\expandafter\def\csname PY@tok@mi\endcsname{\def\PY@tc##1{\textcolor[rgb]{0.40,0.40,0.40}{##1}}}
\expandafter\def\csname PY@tok@il\endcsname{\def\PY@tc##1{\textcolor[rgb]{0.40,0.40,0.40}{##1}}}
\expandafter\def\csname PY@tok@mo\endcsname{\def\PY@tc##1{\textcolor[rgb]{0.40,0.40,0.40}{##1}}}
\expandafter\def\csname PY@tok@ch\endcsname{\let\PY@it=\textit\def\PY@tc##1{\textcolor[rgb]{0.25,0.50,0.50}{##1}}}
\expandafter\def\csname PY@tok@cm\endcsname{\let\PY@it=\textit\def\PY@tc##1{\textcolor[rgb]{0.25,0.50,0.50}{##1}}}
\expandafter\def\csname PY@tok@cpf\endcsname{\let\PY@it=\textit\def\PY@tc##1{\textcolor[rgb]{0.25,0.50,0.50}{##1}}}
\expandafter\def\csname PY@tok@c1\endcsname{\let\PY@it=\textit\def\PY@tc##1{\textcolor[rgb]{0.25,0.50,0.50}{##1}}}
\expandafter\def\csname PY@tok@cs\endcsname{\let\PY@it=\textit\def\PY@tc##1{\textcolor[rgb]{0.25,0.50,0.50}{##1}}}

\def\PYZbs{\char`\\}
\def\PYZus{\char`\_}
\def\PYZob{\char`\{}
\def\PYZcb{\char`\}}
\def\PYZca{\char`\^}
\def\PYZam{\char`\&}
\def\PYZlt{\char`\<}
\def\PYZgt{\char`\>}
\def\PYZsh{\char`\#}
\def\PYZpc{\char`\%}
\def\PYZdl{\char`\$}
\def\PYZhy{\char`\-}
\def\PYZsq{\char`\'}
\def\PYZdq{\char`\"}
\def\PYZti{\char`\~}
% for compatibility with earlier versions
\def\PYZat{@}
\def\PYZlb{[}
\def\PYZrb{]}
\makeatother


    % Exact colors from NB
    \definecolor{incolor}{rgb}{0.0, 0.0, 0.5}
    \definecolor{outcolor}{rgb}{0.545, 0.0, 0.0}



    
    % Prevent overflowing lines due to hard-to-break entities
    \sloppy 
    % Setup hyperref package
    \hypersetup{
      breaklinks=true,  % so long urls are correctly broken across lines
      colorlinks=true,
      urlcolor=urlcolor,
      linkcolor=linkcolor,
      citecolor=citecolor,
      }
    % Slightly bigger margins than the latex defaults
    
    \geometry{verbose,tmargin=1in,bmargin=1in,lmargin=1in,rmargin=1in}
    
    

    \begin{document}
    
    
    \maketitle
    
    

    
    \begin{Verbatim}[commandchars=\\\{\}]
{\color{incolor}In [{\color{incolor}9}]:} \PY{k+kn}{from} \PY{n+nn}{IPython}\PY{n+nn}{.}\PY{n+nn}{display} \PY{k}{import} \PY{n}{HTML}
        \PY{n}{HTML}\PY{p}{(}\PY{l+s+s1}{\PYZsq{}\PYZsq{}\PYZsq{}}\PY{l+s+s1}{\PYZlt{}style\PYZgt{}html, body}\PY{l+s+s1}{\PYZob{}}\PY{l+s+s1}{overflow\PYZhy{}y: visible !important\PYZcb{} .CodeMirror}\PY{l+s+s1}{\PYZob{}}\PY{l+s+s1}{min\PYZhy{}width:105}\PY{l+s+s1}{\PYZpc{}}\PY{l+s+s1}{ !important;\PYZcb{} .rise\PYZhy{}enabled .CodeMirror, .rise\PYZhy{}enabled .output\PYZus{}subarea}\PY{l+s+s1}{\PYZob{}}\PY{l+s+s1}{font\PYZhy{}size:140}\PY{l+s+s1}{\PYZpc{}}\PY{l+s+s1}{; line\PYZhy{}height:1.2; overflow: visible;\PYZcb{} .output\PYZus{}subarea pre}\PY{l+s+si}{\PYZob{}width:110\PYZpc{}\PYZcb{}}\PY{l+s+s1}{\PYZlt{}/style\PYZgt{}}\PY{l+s+s1}{\PYZsq{}\PYZsq{}\PYZsq{}}\PY{p}{)} \PY{c+c1}{\PYZsh{} For slides}
\end{Verbatim}


\begin{Verbatim}[commandchars=\\\{\}]
{\color{outcolor}Out[{\color{outcolor}9}]:} <IPython.core.display.HTML object>
\end{Verbatim}
            
    \section{Foundations of Data Mining: Assignment
1}\label{foundations-of-data-mining-assignment-1}

Please complete all assignments in this notebook. You should submit this
notebook, as well as a PDF version (See File \textgreater{} Download
as).

    \begin{Verbatim}[commandchars=\\\{\}]
{\color{incolor}In [{\color{incolor}10}]:} \PY{c+c1}{\PYZsh{} Please fill in your names here}
         \PY{n}{NAME\PYZus{}STUDENT\PYZus{}1} \PY{o}{=} \PY{l+s+s2}{\PYZdq{}}\PY{l+s+s2}{Bram van der Pol}\PY{l+s+s2}{\PYZdq{}}
         \PY{n}{NAME\PYZus{}STUDENT\PYZus{}2} \PY{o}{=} \PY{l+s+s2}{\PYZdq{}}\PY{l+s+s2}{Joris van der Heijden}\PY{l+s+s2}{\PYZdq{}}
\end{Verbatim}


    \begin{Verbatim}[commandchars=\\\{\}]
{\color{incolor}In [{\color{incolor}11}]:} \PY{o}{\PYZpc{}}\PY{k}{matplotlib} inline
         \PY{k+kn}{from} \PY{n+nn}{preamble} \PY{k}{import} \PY{o}{*}
         \PY{n}{plt}\PY{o}{.}\PY{n}{rcParams}\PY{p}{[}\PY{l+s+s1}{\PYZsq{}}\PY{l+s+s1}{savefig.dpi}\PY{l+s+s1}{\PYZsq{}}\PY{p}{]} \PY{o}{=} \PY{l+m+mi}{100} \PY{c+c1}{\PYZsh{} This controls the size of your figures}
         \PY{c+c1}{\PYZsh{} Comment out and restart notebook if you only want the last output of each cell.}
         \PY{n}{InteractiveShell}\PY{o}{.}\PY{n}{ast\PYZus{}node\PYZus{}interactivity} \PY{o}{=} \PY{l+s+s2}{\PYZdq{}}\PY{l+s+s2}{all}\PY{l+s+s2}{\PYZdq{}}
\end{Verbatim}


    \subsection{MoneyBall (5 points,
1+2+1+1)}\label{moneyball-5-points-1211}

In the early 2000s, 2 baseball scouts completely changed the game of
baseball by analysing the available data about baseball players and
hiring the best ones. The
\href{https://www.openml.org/d/41021}{MoneyBall dataset} contains this
data (click the link for more details). The goal is to accurately
predict the number of 'runs' each player can score.

    \begin{Verbatim}[commandchars=\\\{\}]
{\color{incolor}In [{\color{incolor}168}]:} \PY{n}{moneyball} \PY{o}{=} \PY{n}{oml}\PY{o}{.}\PY{n}{datasets}\PY{o}{.}\PY{n}{get\PYZus{}dataset}\PY{p}{(}\PY{l+m+mi}{41021}\PY{p}{)} \PY{c+c1}{\PYZsh{} Download MoneyBall data}
          \PY{c+c1}{\PYZsh{} Get the predictors X and the target y}
          \PY{n}{X}\PY{p}{,} \PY{n}{y}\PY{p}{,} \PY{n}{attribute\PYZus{}names} \PY{o}{=} \PY{n}{moneyball}\PY{o}{.}\PY{n}{get\PYZus{}data}\PY{p}{(}\PY{n}{target}\PY{o}{=}\PY{n}{moneyball}\PY{o}{.}\PY{n}{default\PYZus{}target\PYZus{}attribute}\PY{p}{,} \PY{n}{return\PYZus{}attribute\PYZus{}names}\PY{o}{=}\PY{k+kc}{True}\PY{p}{)}
          \PY{c+c1}{\PYZsh{} Describe the data with pandas, just to get an overview}
          \PY{c+c1}{\PYZsh{}ballframe = pd.DataFrame(X, columns=attribute\PYZus{}names)}
          \PY{c+c1}{\PYZsh{}ballframe.describe()}
          
          \PY{k+kn}{import} \PY{n+nn}{matplotlib}
          \PY{c+c1}{\PYZsh{}n, bins, patches = ax.hist(X, num\PYZus{}bins, normed=1)}
          
          \PY{l+s+sd}{\PYZsq{}\PYZsq{}\PYZsq{}}
          \PY{l+s+sd}{for index in range(len(X.T)):}
          \PY{l+s+sd}{    print(index)}
          \PY{l+s+sd}{    print(X[:,index])}
          \PY{l+s+sd}{    column=X[:,index]}
          \PY{l+s+sd}{    Filtered\PYZus{}column=column[\PYZti{}np.isnan(column)]}
          \PY{l+s+sd}{    fig1 = plt.figure()}
          \PY{l+s+sd}{    matplotlib.pyplot.hist(Filtered\PYZus{}column)}
          \PY{l+s+sd}{    matplotlib.pyplot.title(\PYZdq{}Index \PYZdq{}+str(index),fontweight=\PYZsq{}bold\PYZsq{},fontsize=15)}
          \PY{l+s+sd}{ \PYZsq{}\PYZsq{}\PYZsq{}}   
          
          \PY{k}{for} \PY{n}{index} \PY{o+ow}{in} \PY{n+nb}{range}\PY{p}{(}\PY{n+nb}{len}\PY{p}{(}\PY{n}{X}\PY{o}{.}\PY{n}{T}\PY{p}{)}\PY{p}{)}\PY{p}{:}
              \PY{c+c1}{\PYZsh{}if (index \PYZlt{} 8):}
              \PY{n+nb}{print}\PY{p}{(}\PY{n}{index}\PY{p}{)}
              \PY{n}{column}\PY{o}{=}\PY{n}{X}\PY{p}{[}\PY{p}{:}\PY{p}{,}\PY{n}{index}\PY{p}{]}
              \PY{n}{Filtered\PYZus{}column1}\PY{o}{=}\PY{n}{column}\PY{p}{[}\PY{o}{\PYZti{}}\PY{n}{np}\PY{o}{.}\PY{n}{isnan}\PY{p}{(}\PY{n}{column}\PY{p}{)}\PY{p}{]}
                  
              \PY{n}{Filtered\PYZus{}column2} \PY{o}{=} \PY{p}{[}\PY{n}{np}\PY{o}{.}\PY{n}{nan} \PY{k}{if} \PY{n}{np}\PY{o}{.}\PY{n}{isnan}\PY{p}{(}\PY{n}{x}\PY{p}{)} \PY{k}{else} \PY{n}{x} \PY{k}{for} \PY{n}{x} \PY{o+ow}{in} \PY{n}{column}\PY{p}{]}
                  
              \PY{n}{fig1} \PY{o}{=} \PY{n}{plt}\PY{o}{.}\PY{n}{figure}\PY{p}{(}\PY{p}{)}
              \PY{n}{matplotlib}\PY{o}{.}\PY{n}{pyplot}\PY{o}{.}\PY{n}{subplot}\PY{p}{(}\PY{l+m+mi}{1}\PY{p}{,}\PY{l+m+mi}{2}\PY{p}{,}\PY{l+m+mi}{1}\PY{p}{)}
              \PY{n}{matplotlib}\PY{o}{.}\PY{n}{pyplot}\PY{o}{.}\PY{n}{hist}\PY{p}{(}\PY{n}{Filtered\PYZus{}column1}\PY{p}{)}
              \PY{n}{matplotlib}\PY{o}{.}\PY{n}{pyplot}\PY{o}{.}\PY{n}{title}\PY{p}{(}\PY{l+s+s1}{\PYZsq{}}\PY{l+s+s1}{Histogram}\PY{l+s+s1}{\PYZsq{}}\PY{p}{,}\PY{n}{fontweight}\PY{o}{=}\PY{l+s+s1}{\PYZsq{}}\PY{l+s+s1}{bold}\PY{l+s+s1}{\PYZsq{}}\PY{p}{,}\PY{n}{fontsize}\PY{o}{=}\PY{l+m+mi}{15}\PY{p}{)}
              
              
              \PY{n}{matplotlib}\PY{o}{.}\PY{n}{pyplot}\PY{o}{.}\PY{n}{subplot}\PY{p}{(}\PY{l+m+mi}{1}\PY{p}{,}\PY{l+m+mi}{2}\PY{p}{,}\PY{l+m+mi}{2}\PY{p}{)}
              \PY{n}{matplotlib}\PY{o}{.}\PY{n}{pyplot}\PY{o}{.}\PY{n}{scatter}\PY{p}{(}\PY{n}{Filtered\PYZus{}column2}\PY{p}{,}\PY{n}{y}\PY{p}{)}
              \PY{n}{matplotlib}\PY{o}{.}\PY{n}{pyplot}\PY{o}{.}\PY{n}{title}\PY{p}{(}\PY{l+s+s1}{\PYZsq{}}\PY{l+s+s1}{Scatter}\PY{l+s+s1}{\PYZsq{}}\PY{p}{,}\PY{n}{fontweight}\PY{o}{=}\PY{l+s+s1}{\PYZsq{}}\PY{l+s+s1}{bold}\PY{l+s+s1}{\PYZsq{}}\PY{p}{,}\PY{n}{fontsize}\PY{o}{=}\PY{l+m+mi}{15}\PY{p}{)}
              \PY{n}{matplotlib}\PY{o}{.}\PY{n}{pyplot}\PY{o}{.}\PY{n}{xlabel}\PY{p}{(}\PY{l+s+s1}{\PYZsq{}}\PY{l+s+s1}{X}\PY{l+s+s1}{\PYZsq{}}\PY{p}{)}
              \PY{n}{matplotlib}\PY{o}{.}\PY{n}{pyplot}\PY{o}{.}\PY{n}{ylabel}\PY{p}{(}\PY{l+s+s1}{\PYZsq{}}\PY{l+s+s1}{y}\PY{l+s+s1}{\PYZsq{}}\PY{p}{)}
              
              \PY{n}{matplotlib}\PY{o}{.}\PY{n}{pyplot}\PY{o}{.}\PY{n}{suptitle}\PY{p}{(}\PY{n}{attribute\PYZus{}names}\PY{p}{[}\PY{n}{index}\PY{p}{]}\PY{o}{+}\PY{l+s+s2}{\PYZdq{}}\PY{l+s+s2}{ Index }\PY{l+s+s2}{\PYZdq{}}\PY{o}{+}\PY{n+nb}{str}\PY{p}{(}\PY{n}{index}\PY{p}{)} \PY{p}{,}\PY{n}{fontweight}\PY{o}{=}\PY{l+s+s1}{\PYZsq{}}\PY{l+s+s1}{bold}\PY{l+s+s1}{\PYZsq{}}\PY{p}{,}\PY{n}{fontsize}\PY{o}{=}\PY{l+m+mi}{15}\PY{p}{)}
              \PY{n}{matplotlib}\PY{o}{.}\PY{n}{pyplot}\PY{o}{.}\PY{n}{subplots\PYZus{}adjust}\PY{p}{(}\PY{n}{left}\PY{o}{=}\PY{l+m+mf}{0.2}\PY{p}{,}\PY{n}{wspace}\PY{o}{=}\PY{l+m+mf}{0.8}\PY{p}{,}\PY{n}{top}\PY{o}{=}\PY{l+m+mf}{0.8}\PY{p}{)}
              
\end{Verbatim}


\begin{Verbatim}[commandchars=\\\{\}]
{\color{outcolor}Out[{\color{outcolor}168}]:} '\textbackslash{}nfor index in range(len(X.T)):\textbackslash{}n    print(index)\textbackslash{}n    print(X[:,index])\textbackslash{}n    column=X[:,index]\textbackslash{}n    Filtered\_column=column[\textasciitilde{}np.isnan(column)]\textbackslash{}n    fig1 = plt.figure()\textbackslash{}n    matplotlib.pyplot.hist(Filtered\_column)\textbackslash{}n    matplotlib.pyplot.title("Index "+str(index),fontweight=\textbackslash{}'bold\textbackslash{}',fontsize=15)\textbackslash{}n '
\end{Verbatim}
            
    \begin{Verbatim}[commandchars=\\\{\}]
0

    \end{Verbatim}

\begin{Verbatim}[commandchars=\\\{\}]
{\color{outcolor}Out[{\color{outcolor}168}]:} <matplotlib.axes.\_subplots.AxesSubplot at 0x1aa784009b0>
\end{Verbatim}
            
\begin{Verbatim}[commandchars=\\\{\}]
{\color{outcolor}Out[{\color{outcolor}168}]:} (array([152., 188., 152.,  98., 141., 175., 132.,  95.,  78.,  21.]),
           array([ 0. ,  3.8,  7.6, 11.4, 15.2, 19. , 22.8, 26.6, 30.4, 34.2, 38. ]),
           <a list of 10 Patch objects>)
\end{Verbatim}
            
\begin{Verbatim}[commandchars=\\\{\}]
{\color{outcolor}Out[{\color{outcolor}168}]:} Text(0.5,1,'Histogram')
\end{Verbatim}
            
\begin{Verbatim}[commandchars=\\\{\}]
{\color{outcolor}Out[{\color{outcolor}168}]:} <matplotlib.axes.\_subplots.AxesSubplot at 0x1aa7c808c50>
\end{Verbatim}
            
\begin{Verbatim}[commandchars=\\\{\}]
{\color{outcolor}Out[{\color{outcolor}168}]:} <matplotlib.collections.PathCollection at 0x1aa7b1f6cc0>
\end{Verbatim}
            
\begin{Verbatim}[commandchars=\\\{\}]
{\color{outcolor}Out[{\color{outcolor}168}]:} Text(0.5,1,'Scatter')
\end{Verbatim}
            
\begin{Verbatim}[commandchars=\\\{\}]
{\color{outcolor}Out[{\color{outcolor}168}]:} Text(0.5,0,'X')
\end{Verbatim}
            
\begin{Verbatim}[commandchars=\\\{\}]
{\color{outcolor}Out[{\color{outcolor}168}]:} Text(0,0.5,'y')
\end{Verbatim}
            
\begin{Verbatim}[commandchars=\\\{\}]
{\color{outcolor}Out[{\color{outcolor}168}]:} Text(0.5,0.98,'Team Index 0')
\end{Verbatim}
            
    \begin{Verbatim}[commandchars=\\\{\}]
1

    \end{Verbatim}

\begin{Verbatim}[commandchars=\\\{\}]
{\color{outcolor}Out[{\color{outcolor}168}]:} <matplotlib.axes.\_subplots.AxesSubplot at 0x1aa7b1f6f28>
\end{Verbatim}
            
\begin{Verbatim}[commandchars=\\\{\}]
{\color{outcolor}Out[{\color{outcolor}168}]:} (array([616.,   0.,   0.,   0.,   0.,   0.,   0.,   0.,   0., 616.]),
           array([0. , 0.1, 0.2, 0.3, 0.4, 0.5, 0.6, 0.7, 0.8, 0.9, 1. ]),
           <a list of 10 Patch objects>)
\end{Verbatim}
            
\begin{Verbatim}[commandchars=\\\{\}]
{\color{outcolor}Out[{\color{outcolor}168}]:} Text(0.5,1,'Histogram')
\end{Verbatim}
            
\begin{Verbatim}[commandchars=\\\{\}]
{\color{outcolor}Out[{\color{outcolor}168}]:} <matplotlib.axes.\_subplots.AxesSubplot at 0x1aa79ba2f60>
\end{Verbatim}
            
\begin{Verbatim}[commandchars=\\\{\}]
{\color{outcolor}Out[{\color{outcolor}168}]:} <matplotlib.collections.PathCollection at 0x1aa7cd04cf8>
\end{Verbatim}
            
\begin{Verbatim}[commandchars=\\\{\}]
{\color{outcolor}Out[{\color{outcolor}168}]:} Text(0.5,1,'Scatter')
\end{Verbatim}
            
\begin{Verbatim}[commandchars=\\\{\}]
{\color{outcolor}Out[{\color{outcolor}168}]:} Text(0.5,0,'X')
\end{Verbatim}
            
\begin{Verbatim}[commandchars=\\\{\}]
{\color{outcolor}Out[{\color{outcolor}168}]:} Text(0,0.5,'y')
\end{Verbatim}
            
\begin{Verbatim}[commandchars=\\\{\}]
{\color{outcolor}Out[{\color{outcolor}168}]:} Text(0.5,0.98,'League Index 1')
\end{Verbatim}
            
    \begin{Verbatim}[commandchars=\\\{\}]
2

    \end{Verbatim}

\begin{Verbatim}[commandchars=\\\{\}]
{\color{outcolor}Out[{\color{outcolor}168}]:} <matplotlib.axes.\_subplots.AxesSubplot at 0x1aa7cd041d0>
\end{Verbatim}
            
\begin{Verbatim}[commandchars=\\\{\}]
{\color{outcolor}Out[{\color{outcolor}168}]:} (array([100., 112.,  96., 104., 130., 130.,  82., 148., 150., 180.]),
           array([1962., 1967., 1972., 1977., 1982., 1987., 1992., 1997., 2002.,
                  2007., 2012.]),
           <a list of 10 Patch objects>)
\end{Verbatim}
            
\begin{Verbatim}[commandchars=\\\{\}]
{\color{outcolor}Out[{\color{outcolor}168}]:} Text(0.5,1,'Histogram')
\end{Verbatim}
            
\begin{Verbatim}[commandchars=\\\{\}]
{\color{outcolor}Out[{\color{outcolor}168}]:} <matplotlib.axes.\_subplots.AxesSubplot at 0x1aa76689a58>
\end{Verbatim}
            
\begin{Verbatim}[commandchars=\\\{\}]
{\color{outcolor}Out[{\color{outcolor}168}]:} <matplotlib.collections.PathCollection at 0x1aa76d53320>
\end{Verbatim}
            
\begin{Verbatim}[commandchars=\\\{\}]
{\color{outcolor}Out[{\color{outcolor}168}]:} Text(0.5,1,'Scatter')
\end{Verbatim}
            
\begin{Verbatim}[commandchars=\\\{\}]
{\color{outcolor}Out[{\color{outcolor}168}]:} Text(0.5,0,'X')
\end{Verbatim}
            
\begin{Verbatim}[commandchars=\\\{\}]
{\color{outcolor}Out[{\color{outcolor}168}]:} Text(0,0.5,'y')
\end{Verbatim}
            
\begin{Verbatim}[commandchars=\\\{\}]
{\color{outcolor}Out[{\color{outcolor}168}]:} Text(0.5,0.98,'Year Index 2')
\end{Verbatim}
            
    \begin{Verbatim}[commandchars=\\\{\}]
3

    \end{Verbatim}

\begin{Verbatim}[commandchars=\\\{\}]
{\color{outcolor}Out[{\color{outcolor}168}]:} <matplotlib.axes.\_subplots.AxesSubplot at 0x1aa783a9be0>
\end{Verbatim}
            
\begin{Verbatim}[commandchars=\\\{\}]
{\color{outcolor}Out[{\color{outcolor}168}]:} (array([ 20., 110., 239., 327., 270., 163.,  77.,  24.,   1.,   1.]),
           array([ 472. ,  535.1,  598.2,  661.3,  724.4,  787.5,  850.6,  913.7,
                   976.8, 1039.9, 1103. ]),
           <a list of 10 Patch objects>)
\end{Verbatim}
            
\begin{Verbatim}[commandchars=\\\{\}]
{\color{outcolor}Out[{\color{outcolor}168}]:} Text(0.5,1,'Histogram')
\end{Verbatim}
            
\begin{Verbatim}[commandchars=\\\{\}]
{\color{outcolor}Out[{\color{outcolor}168}]:} <matplotlib.axes.\_subplots.AxesSubplot at 0x1aa7b16ba90>
\end{Verbatim}
            
\begin{Verbatim}[commandchars=\\\{\}]
{\color{outcolor}Out[{\color{outcolor}168}]:} <matplotlib.collections.PathCollection at 0x1aa7cb0ce48>
\end{Verbatim}
            
\begin{Verbatim}[commandchars=\\\{\}]
{\color{outcolor}Out[{\color{outcolor}168}]:} Text(0.5,1,'Scatter')
\end{Verbatim}
            
\begin{Verbatim}[commandchars=\\\{\}]
{\color{outcolor}Out[{\color{outcolor}168}]:} Text(0.5,0,'X')
\end{Verbatim}
            
\begin{Verbatim}[commandchars=\\\{\}]
{\color{outcolor}Out[{\color{outcolor}168}]:} Text(0,0.5,'y')
\end{Verbatim}
            
\begin{Verbatim}[commandchars=\\\{\}]
{\color{outcolor}Out[{\color{outcolor}168}]:} Text(0.5,0.98,'RA Index 3')
\end{Verbatim}
            
    \begin{Verbatim}[commandchars=\\\{\}]
4

    \end{Verbatim}

\begin{Verbatim}[commandchars=\\\{\}]
{\color{outcolor}Out[{\color{outcolor}168}]:} <matplotlib.axes.\_subplots.AxesSubplot at 0x1aa7cb0cef0>
\end{Verbatim}
            
\begin{Verbatim}[commandchars=\\\{\}]
{\color{outcolor}Out[{\color{outcolor}168}]:} (array([  2.,  17.,  50., 167., 243., 291., 286., 134.,  39.,   3.]),
           array([ 40. ,  47.6,  55.2,  62.8,  70.4,  78. ,  85.6,  93.2, 100.8,
                  108.4, 116. ]),
           <a list of 10 Patch objects>)
\end{Verbatim}
            
\begin{Verbatim}[commandchars=\\\{\}]
{\color{outcolor}Out[{\color{outcolor}168}]:} Text(0.5,1,'Histogram')
\end{Verbatim}
            
\begin{Verbatim}[commandchars=\\\{\}]
{\color{outcolor}Out[{\color{outcolor}168}]:} <matplotlib.axes.\_subplots.AxesSubplot at 0x1aa76be2d68>
\end{Verbatim}
            
\begin{Verbatim}[commandchars=\\\{\}]
{\color{outcolor}Out[{\color{outcolor}168}]:} <matplotlib.collections.PathCollection at 0x1aa769ed978>
\end{Verbatim}
            
\begin{Verbatim}[commandchars=\\\{\}]
{\color{outcolor}Out[{\color{outcolor}168}]:} Text(0.5,1,'Scatter')
\end{Verbatim}
            
\begin{Verbatim}[commandchars=\\\{\}]
{\color{outcolor}Out[{\color{outcolor}168}]:} Text(0.5,0,'X')
\end{Verbatim}
            
\begin{Verbatim}[commandchars=\\\{\}]
{\color{outcolor}Out[{\color{outcolor}168}]:} Text(0,0.5,'y')
\end{Verbatim}
            
\begin{Verbatim}[commandchars=\\\{\}]
{\color{outcolor}Out[{\color{outcolor}168}]:} Text(0.5,0.98,'W Index 4')
\end{Verbatim}
            
    \begin{Verbatim}[commandchars=\\\{\}]
5

    \end{Verbatim}

\begin{Verbatim}[commandchars=\\\{\}]
{\color{outcolor}Out[{\color{outcolor}168}]:} <matplotlib.axes.\_subplots.AxesSubplot at 0x1aa7cb564e0>
\end{Verbatim}
            
\begin{Verbatim}[commandchars=\\\{\}]
{\color{outcolor}Out[{\color{outcolor}168}]:} (array([  7.,  22.,  64., 193., 322., 266., 225.,  81.,  45.,   7.]),
           array([0.277, 0.287, 0.296, 0.306, 0.315, 0.325, 0.335, 0.344, 0.354,
                  0.363, 0.373]),
           <a list of 10 Patch objects>)
\end{Verbatim}
            
\begin{Verbatim}[commandchars=\\\{\}]
{\color{outcolor}Out[{\color{outcolor}168}]:} Text(0.5,1,'Histogram')
\end{Verbatim}
            
\begin{Verbatim}[commandchars=\\\{\}]
{\color{outcolor}Out[{\color{outcolor}168}]:} <matplotlib.axes.\_subplots.AxesSubplot at 0x1aa7994ef28>
\end{Verbatim}
            
\begin{Verbatim}[commandchars=\\\{\}]
{\color{outcolor}Out[{\color{outcolor}168}]:} <matplotlib.collections.PathCollection at 0x1aa7c860eb8>
\end{Verbatim}
            
\begin{Verbatim}[commandchars=\\\{\}]
{\color{outcolor}Out[{\color{outcolor}168}]:} Text(0.5,1,'Scatter')
\end{Verbatim}
            
\begin{Verbatim}[commandchars=\\\{\}]
{\color{outcolor}Out[{\color{outcolor}168}]:} Text(0.5,0,'X')
\end{Verbatim}
            
\begin{Verbatim}[commandchars=\\\{\}]
{\color{outcolor}Out[{\color{outcolor}168}]:} Text(0,0.5,'y')
\end{Verbatim}
            
\begin{Verbatim}[commandchars=\\\{\}]
{\color{outcolor}Out[{\color{outcolor}168}]:} Text(0.5,0.98,'OBP Index 5')
\end{Verbatim}
            
    \begin{Verbatim}[commandchars=\\\{\}]
6

    \end{Verbatim}

\begin{Verbatim}[commandchars=\\\{\}]
{\color{outcolor}Out[{\color{outcolor}168}]:} <matplotlib.axes.\_subplots.AxesSubplot at 0x1aa79a22940>
\end{Verbatim}
            
\begin{Verbatim}[commandchars=\\\{\}]
{\color{outcolor}Out[{\color{outcolor}168}]:} (array([ 13.,  31., 100., 186., 276., 258., 202.,  93.,  57.,  16.]),
           array([0.301, 0.32 , 0.339, 0.358, 0.377, 0.396, 0.415, 0.434, 0.453,
                  0.472, 0.491]),
           <a list of 10 Patch objects>)
\end{Verbatim}
            
\begin{Verbatim}[commandchars=\\\{\}]
{\color{outcolor}Out[{\color{outcolor}168}]:} Text(0.5,1,'Histogram')
\end{Verbatim}
            
\begin{Verbatim}[commandchars=\\\{\}]
{\color{outcolor}Out[{\color{outcolor}168}]:} <matplotlib.axes.\_subplots.AxesSubplot at 0x1aa7c81ff28>
\end{Verbatim}
            
\begin{Verbatim}[commandchars=\\\{\}]
{\color{outcolor}Out[{\color{outcolor}168}]:} <matplotlib.collections.PathCollection at 0x1aa7ce6b550>
\end{Verbatim}
            
\begin{Verbatim}[commandchars=\\\{\}]
{\color{outcolor}Out[{\color{outcolor}168}]:} Text(0.5,1,'Scatter')
\end{Verbatim}
            
\begin{Verbatim}[commandchars=\\\{\}]
{\color{outcolor}Out[{\color{outcolor}168}]:} Text(0.5,0,'X')
\end{Verbatim}
            
\begin{Verbatim}[commandchars=\\\{\}]
{\color{outcolor}Out[{\color{outcolor}168}]:} Text(0,0.5,'y')
\end{Verbatim}
            
\begin{Verbatim}[commandchars=\\\{\}]
{\color{outcolor}Out[{\color{outcolor}168}]:} Text(0.5,0.98,'SLG Index 6')
\end{Verbatim}
            
    \begin{Verbatim}[commandchars=\\\{\}]
7

    \end{Verbatim}

\begin{Verbatim}[commandchars=\\\{\}]
{\color{outcolor}Out[{\color{outcolor}168}]:} <matplotlib.axes.\_subplots.AxesSubplot at 0x1aa79a529b0>
\end{Verbatim}
            
\begin{Verbatim}[commandchars=\\\{\}]
{\color{outcolor}Out[{\color{outcolor}168}]:} (array([  4.,  14.,  41., 102., 242., 329., 229., 186.,  58.,  27.]),
           array([0.214, 0.222, 0.23 , 0.238, 0.246, 0.254, 0.262, 0.27 , 0.278,
                  0.286, 0.294]),
           <a list of 10 Patch objects>)
\end{Verbatim}
            
\begin{Verbatim}[commandchars=\\\{\}]
{\color{outcolor}Out[{\color{outcolor}168}]:} Text(0.5,1,'Histogram')
\end{Verbatim}
            
\begin{Verbatim}[commandchars=\\\{\}]
{\color{outcolor}Out[{\color{outcolor}168}]:} <matplotlib.axes.\_subplots.AxesSubplot at 0x1aa7cb43668>
\end{Verbatim}
            
\begin{Verbatim}[commandchars=\\\{\}]
{\color{outcolor}Out[{\color{outcolor}168}]:} <matplotlib.collections.PathCollection at 0x1aa77d705f8>
\end{Verbatim}
            
\begin{Verbatim}[commandchars=\\\{\}]
{\color{outcolor}Out[{\color{outcolor}168}]:} Text(0.5,1,'Scatter')
\end{Verbatim}
            
\begin{Verbatim}[commandchars=\\\{\}]
{\color{outcolor}Out[{\color{outcolor}168}]:} Text(0.5,0,'X')
\end{Verbatim}
            
\begin{Verbatim}[commandchars=\\\{\}]
{\color{outcolor}Out[{\color{outcolor}168}]:} Text(0,0.5,'y')
\end{Verbatim}
            
\begin{Verbatim}[commandchars=\\\{\}]
{\color{outcolor}Out[{\color{outcolor}168}]:} Text(0.5,0.98,'BA Index 7')
\end{Verbatim}
            
    \begin{Verbatim}[commandchars=\\\{\}]
8

    \end{Verbatim}

\begin{Verbatim}[commandchars=\\\{\}]
{\color{outcolor}Out[{\color{outcolor}168}]:} <matplotlib.axes.\_subplots.AxesSubplot at 0x1aa7cd71208>
\end{Verbatim}
            
\begin{Verbatim}[commandchars=\\\{\}]
{\color{outcolor}Out[{\color{outcolor}168}]:} (array([988.,   0.,   0.,   0.,   0.,   0.,   0.,   0.,   0., 244.]),
           array([0. , 0.1, 0.2, 0.3, 0.4, 0.5, 0.6, 0.7, 0.8, 0.9, 1. ]),
           <a list of 10 Patch objects>)
\end{Verbatim}
            
\begin{Verbatim}[commandchars=\\\{\}]
{\color{outcolor}Out[{\color{outcolor}168}]:} Text(0.5,1,'Histogram')
\end{Verbatim}
            
\begin{Verbatim}[commandchars=\\\{\}]
{\color{outcolor}Out[{\color{outcolor}168}]:} <matplotlib.axes.\_subplots.AxesSubplot at 0x1aa79b2b3c8>
\end{Verbatim}
            
\begin{Verbatim}[commandchars=\\\{\}]
{\color{outcolor}Out[{\color{outcolor}168}]:} <matplotlib.collections.PathCollection at 0x1aa76cd4ac8>
\end{Verbatim}
            
\begin{Verbatim}[commandchars=\\\{\}]
{\color{outcolor}Out[{\color{outcolor}168}]:} Text(0.5,1,'Scatter')
\end{Verbatim}
            
\begin{Verbatim}[commandchars=\\\{\}]
{\color{outcolor}Out[{\color{outcolor}168}]:} Text(0.5,0,'X')
\end{Verbatim}
            
\begin{Verbatim}[commandchars=\\\{\}]
{\color{outcolor}Out[{\color{outcolor}168}]:} Text(0,0.5,'y')
\end{Verbatim}
            
\begin{Verbatim}[commandchars=\\\{\}]
{\color{outcolor}Out[{\color{outcolor}168}]:} Text(0.5,0.98,'Playoffs Index 8')
\end{Verbatim}
            
    \begin{Verbatim}[commandchars=\\\{\}]
9

    \end{Verbatim}

\begin{Verbatim}[commandchars=\\\{\}]
{\color{outcolor}Out[{\color{outcolor}168}]:} <matplotlib.axes.\_subplots.AxesSubplot at 0x1aa798e0160>
\end{Verbatim}
            
\begin{Verbatim}[commandchars=\\\{\}]
{\color{outcolor}Out[{\color{outcolor}168}]:} (array([52., 53., 44.,  0., 44., 21.,  0., 20.,  9.,  1.]),
           array([0. , 0.7, 1.4, 2.1, 2.8, 3.5, 4.2, 4.9, 5.6, 6.3, 7. ]),
           <a list of 10 Patch objects>)
\end{Verbatim}
            
\begin{Verbatim}[commandchars=\\\{\}]
{\color{outcolor}Out[{\color{outcolor}168}]:} Text(0.5,1,'Histogram')
\end{Verbatim}
            
\begin{Verbatim}[commandchars=\\\{\}]
{\color{outcolor}Out[{\color{outcolor}168}]:} <matplotlib.axes.\_subplots.AxesSubplot at 0x1aa78615b70>
\end{Verbatim}
            
\begin{Verbatim}[commandchars=\\\{\}]
{\color{outcolor}Out[{\color{outcolor}168}]:} <matplotlib.collections.PathCollection at 0x1aa7d4d00f0>
\end{Verbatim}
            
\begin{Verbatim}[commandchars=\\\{\}]
{\color{outcolor}Out[{\color{outcolor}168}]:} Text(0.5,1,'Scatter')
\end{Verbatim}
            
\begin{Verbatim}[commandchars=\\\{\}]
{\color{outcolor}Out[{\color{outcolor}168}]:} Text(0.5,0,'X')
\end{Verbatim}
            
\begin{Verbatim}[commandchars=\\\{\}]
{\color{outcolor}Out[{\color{outcolor}168}]:} Text(0,0.5,'y')
\end{Verbatim}
            
\begin{Verbatim}[commandchars=\\\{\}]
{\color{outcolor}Out[{\color{outcolor}168}]:} Text(0.5,0.98,'RankSeason Index 9')
\end{Verbatim}
            
    \begin{Verbatim}[commandchars=\\\{\}]
10

    \end{Verbatim}

\begin{Verbatim}[commandchars=\\\{\}]
{\color{outcolor}Out[{\color{outcolor}168}]:} <matplotlib.axes.\_subplots.AxesSubplot at 0x1aa7d5700b8>
\end{Verbatim}
            
\begin{Verbatim}[commandchars=\\\{\}]
{\color{outcolor}Out[{\color{outcolor}168}]:} (array([47.,  0., 47.,  0.,  0., 80.,  0., 68.,  0.,  2.]),
           array([0. , 0.4, 0.8, 1.2, 1.6, 2. , 2.4, 2.8, 3.2, 3.6, 4. ]),
           <a list of 10 Patch objects>)
\end{Verbatim}
            
\begin{Verbatim}[commandchars=\\\{\}]
{\color{outcolor}Out[{\color{outcolor}168}]:} Text(0.5,1,'Histogram')
\end{Verbatim}
            
\begin{Verbatim}[commandchars=\\\{\}]
{\color{outcolor}Out[{\color{outcolor}168}]:} <matplotlib.axes.\_subplots.AxesSubplot at 0x1aa76d27668>
\end{Verbatim}
            
\begin{Verbatim}[commandchars=\\\{\}]
{\color{outcolor}Out[{\color{outcolor}168}]:} <matplotlib.collections.PathCollection at 0x1aa7cb3def0>
\end{Verbatim}
            
\begin{Verbatim}[commandchars=\\\{\}]
{\color{outcolor}Out[{\color{outcolor}168}]:} Text(0.5,1,'Scatter')
\end{Verbatim}
            
\begin{Verbatim}[commandchars=\\\{\}]
{\color{outcolor}Out[{\color{outcolor}168}]:} Text(0.5,0,'X')
\end{Verbatim}
            
\begin{Verbatim}[commandchars=\\\{\}]
{\color{outcolor}Out[{\color{outcolor}168}]:} Text(0,0.5,'y')
\end{Verbatim}
            
\begin{Verbatim}[commandchars=\\\{\}]
{\color{outcolor}Out[{\color{outcolor}168}]:} Text(0.5,0.98,'RankPlayoffs Index 10')
\end{Verbatim}
            
    \begin{Verbatim}[commandchars=\\\{\}]
11

    \end{Verbatim}

\begin{Verbatim}[commandchars=\\\{\}]
{\color{outcolor}Out[{\color{outcolor}168}]:} <matplotlib.axes.\_subplots.AxesSubplot at 0x1aa7c799940>
\end{Verbatim}
            
\begin{Verbatim}[commandchars=\\\{\}]
{\color{outcolor}Out[{\color{outcolor}168}]:} (array([  1.,  10.,  23.,   0., 139., 954.,   0.,  93.,  10.,   2.]),
           array([0. , 0.7, 1.4, 2.1, 2.8, 3.5, 4.2, 4.9, 5.6, 6.3, 7. ]),
           <a list of 10 Patch objects>)
\end{Verbatim}
            
\begin{Verbatim}[commandchars=\\\{\}]
{\color{outcolor}Out[{\color{outcolor}168}]:} Text(0.5,1,'Histogram')
\end{Verbatim}
            
\begin{Verbatim}[commandchars=\\\{\}]
{\color{outcolor}Out[{\color{outcolor}168}]:} <matplotlib.axes.\_subplots.AxesSubplot at 0x1aa7c507b00>
\end{Verbatim}
            
\begin{Verbatim}[commandchars=\\\{\}]
{\color{outcolor}Out[{\color{outcolor}168}]:} <matplotlib.collections.PathCollection at 0x1aa79d81cf8>
\end{Verbatim}
            
\begin{Verbatim}[commandchars=\\\{\}]
{\color{outcolor}Out[{\color{outcolor}168}]:} Text(0.5,1,'Scatter')
\end{Verbatim}
            
\begin{Verbatim}[commandchars=\\\{\}]
{\color{outcolor}Out[{\color{outcolor}168}]:} Text(0.5,0,'X')
\end{Verbatim}
            
\begin{Verbatim}[commandchars=\\\{\}]
{\color{outcolor}Out[{\color{outcolor}168}]:} Text(0,0.5,'y')
\end{Verbatim}
            
\begin{Verbatim}[commandchars=\\\{\}]
{\color{outcolor}Out[{\color{outcolor}168}]:} Text(0.5,0.98,'G Index 11')
\end{Verbatim}
            
    \begin{Verbatim}[commandchars=\\\{\}]
12

    \end{Verbatim}

\begin{Verbatim}[commandchars=\\\{\}]
{\color{outcolor}Out[{\color{outcolor}168}]:} <matplotlib.axes.\_subplots.AxesSubplot at 0x1aa750bd828>
\end{Verbatim}
            
\begin{Verbatim}[commandchars=\\\{\}]
{\color{outcolor}Out[{\color{outcolor}168}]:} (array([ 4., 33., 64., 89., 94., 74., 41., 15.,  5.,  1.]),
           array([0.294, 0.303, 0.312, 0.321, 0.33 , 0.339, 0.348, 0.357, 0.366,
                  0.375, 0.384]),
           <a list of 10 Patch objects>)
\end{Verbatim}
            
\begin{Verbatim}[commandchars=\\\{\}]
{\color{outcolor}Out[{\color{outcolor}168}]:} Text(0.5,1,'Histogram')
\end{Verbatim}
            
\begin{Verbatim}[commandchars=\\\{\}]
{\color{outcolor}Out[{\color{outcolor}168}]:} <matplotlib.axes.\_subplots.AxesSubplot at 0x1aa7ce93198>
\end{Verbatim}
            
\begin{Verbatim}[commandchars=\\\{\}]
{\color{outcolor}Out[{\color{outcolor}168}]:} <matplotlib.collections.PathCollection at 0x1aa79838a20>
\end{Verbatim}
            
\begin{Verbatim}[commandchars=\\\{\}]
{\color{outcolor}Out[{\color{outcolor}168}]:} Text(0.5,1,'Scatter')
\end{Verbatim}
            
\begin{Verbatim}[commandchars=\\\{\}]
{\color{outcolor}Out[{\color{outcolor}168}]:} Text(0.5,0,'X')
\end{Verbatim}
            
\begin{Verbatim}[commandchars=\\\{\}]
{\color{outcolor}Out[{\color{outcolor}168}]:} Text(0,0.5,'y')
\end{Verbatim}
            
\begin{Verbatim}[commandchars=\\\{\}]
{\color{outcolor}Out[{\color{outcolor}168}]:} Text(0.5,0.98,'OOBP Index 12')
\end{Verbatim}
            
    \begin{Verbatim}[commandchars=\\\{\}]
13

    \end{Verbatim}

\begin{Verbatim}[commandchars=\\\{\}]
{\color{outcolor}Out[{\color{outcolor}168}]:} <matplotlib.axes.\_subplots.AxesSubplot at 0x1aa7caaa390>
\end{Verbatim}
            
\begin{Verbatim}[commandchars=\\\{\}]
{\color{outcolor}Out[{\color{outcolor}168}]:} (array([ 6., 14., 36., 92., 82., 82., 61., 31., 14.,  2.]),
           array([0.346, 0.361, 0.377, 0.392, 0.407, 0.422, 0.438, 0.453, 0.468,
                  0.484, 0.499]),
           <a list of 10 Patch objects>)
\end{Verbatim}
            
\begin{Verbatim}[commandchars=\\\{\}]
{\color{outcolor}Out[{\color{outcolor}168}]:} Text(0.5,1,'Histogram')
\end{Verbatim}
            
\begin{Verbatim}[commandchars=\\\{\}]
{\color{outcolor}Out[{\color{outcolor}168}]:} <matplotlib.axes.\_subplots.AxesSubplot at 0x1aa7cc6bb38>
\end{Verbatim}
            
\begin{Verbatim}[commandchars=\\\{\}]
{\color{outcolor}Out[{\color{outcolor}168}]:} <matplotlib.collections.PathCollection at 0x1aa7ca5d588>
\end{Verbatim}
            
\begin{Verbatim}[commandchars=\\\{\}]
{\color{outcolor}Out[{\color{outcolor}168}]:} Text(0.5,1,'Scatter')
\end{Verbatim}
            
\begin{Verbatim}[commandchars=\\\{\}]
{\color{outcolor}Out[{\color{outcolor}168}]:} Text(0.5,0,'X')
\end{Verbatim}
            
\begin{Verbatim}[commandchars=\\\{\}]
{\color{outcolor}Out[{\color{outcolor}168}]:} Text(0,0.5,'y')
\end{Verbatim}
            
\begin{Verbatim}[commandchars=\\\{\}]
{\color{outcolor}Out[{\color{outcolor}168}]:} Text(0.5,0.98,'OSLG Index 13')
\end{Verbatim}
            
    \begin{center}
    \adjustimage{max size={0.9\linewidth}{0.9\paperheight}}{output_5_141.pdf}
    \end{center}
    { \hspace*{\fill} \\}
    
    \begin{center}
    \adjustimage{max size={0.9\linewidth}{0.9\paperheight}}{output_5_142.pdf}
    \end{center}
    { \hspace*{\fill} \\}
    
    \begin{center}
    \adjustimage{max size={0.9\linewidth}{0.9\paperheight}}{output_5_143.pdf}
    \end{center}
    { \hspace*{\fill} \\}
    
    \begin{center}
    \adjustimage{max size={0.9\linewidth}{0.9\paperheight}}{output_5_144.pdf}
    \end{center}
    { \hspace*{\fill} \\}
    
    \begin{center}
    \adjustimage{max size={0.9\linewidth}{0.9\paperheight}}{output_5_145.pdf}
    \end{center}
    { \hspace*{\fill} \\}
    
    \begin{center}
    \adjustimage{max size={0.9\linewidth}{0.9\paperheight}}{output_5_146.pdf}
    \end{center}
    { \hspace*{\fill} \\}
    
    \begin{center}
    \adjustimage{max size={0.9\linewidth}{0.9\paperheight}}{output_5_147.pdf}
    \end{center}
    { \hspace*{\fill} \\}
    
    \begin{center}
    \adjustimage{max size={0.9\linewidth}{0.9\paperheight}}{output_5_148.pdf}
    \end{center}
    { \hspace*{\fill} \\}
    
    \begin{center}
    \adjustimage{max size={0.9\linewidth}{0.9\paperheight}}{output_5_149.pdf}
    \end{center}
    { \hspace*{\fill} \\}
    
    \begin{center}
    \adjustimage{max size={0.9\linewidth}{0.9\paperheight}}{output_5_150.pdf}
    \end{center}
    { \hspace*{\fill} \\}
    
    \begin{center}
    \adjustimage{max size={0.9\linewidth}{0.9\paperheight}}{output_5_151.pdf}
    \end{center}
    { \hspace*{\fill} \\}
    
    \begin{center}
    \adjustimage{max size={0.9\linewidth}{0.9\paperheight}}{output_5_152.pdf}
    \end{center}
    { \hspace*{\fill} \\}
    
    \begin{center}
    \adjustimage{max size={0.9\linewidth}{0.9\paperheight}}{output_5_153.pdf}
    \end{center}
    { \hspace*{\fill} \\}
    
    \begin{center}
    \adjustimage{max size={0.9\linewidth}{0.9\paperheight}}{output_5_154.pdf}
    \end{center}
    { \hspace*{\fill} \\}
    
    1 . Visually explore the data. Plot the distribution of each feature
(e.g. histograms), as well as the target. Visualize the dependency of
the target on each feature (use a 2d scatter plot). Is there anything
that stands out? Is there something that you think might require special
treatment? - Feel free to create additional plots that help you
understand the data - Only visualize the data, you don't need to change
it (yet)

\textbf{Answers:} The first column of figures shows the histograms of
each column of the data in X. The second column shows the relationship
between y and x. \emph{Is there anything that stands out?} The dataset
containts non real numbers, so the data set must first be cleaned to
visualize it. Beside the NaN the data the figures that do not show a
clear distrubution are: 1, 8, 10 and 11. The scatterplots shows

Is there something that you think might require special treatment?

    2 . Compare all linear regression algorithms that we covered in class
(Linear Regression, Ridge, Lasso and ElasticNet), as well as kNN.
Evaluate using cross-validation and the \(R^2\) score, with the default
parameters. Does scaling the data with StandardScaler help? Provide a
concise but meaningful interpretation of the results. - Preprocess the
data as needed (e.g. are there nominal features that are not ordinal?).
If you don't know how to proceed, remove the feature and continue.

    3 . Do a default, shuffled train-test split and optimize the linear
models for the degree of regularization (\(alpha\)) and choice of
penalty (L1/L2). For Ridge and Lasso, plot a curve showing the effect of
the training and test set performance (\(R^2\)) while increasing the
degree of regularization for different penalties. For ElasticNet, plot a
heatmap \(alpha \times l1\_ratio \rightarrow R^2\) using test set
performance. Report the optimal performance. Again, provide a concise
but meaningful interpretation. What does the regularization do? Can you
get better results? - Think about how you get the L1/L2 loss. This is
not a hyperparameter in regression. - We've seen how to generate such
heatmaps in Lecture 3.

    4 . Visualize the coefficients of the optimized models. Do they agree on
which features are important? Compare the results with the feature
importances returned by a RandomForest. Does it agree with the linear
models? What would look for when scouting for a baseball player?

    \subsection{Nepalese character recognition (5 points,
1+2+2)}\label{nepalese-character-recognition-5-points-122}

The \href{https://www.openml.org/d/40923}{Devnagari-Script dataset}
contains 92,000 images (32x32 pixels) of 46 characters from Devanagari
script. Your goal is to learn to recognize the right letter given the
image.

    \begin{Verbatim}[commandchars=\\\{\}]
{\color{incolor}In [{\color{incolor}63}]:} \PY{n}{devnagari} \PY{o}{=} \PY{n}{oml}\PY{o}{.}\PY{n}{datasets}\PY{o}{.}\PY{n}{get\PYZus{}dataset}\PY{p}{(}\PY{l+m+mi}{40923}\PY{p}{)} \PY{c+c1}{\PYZsh{} Download Devnagari data}
         \PY{c+c1}{\PYZsh{} Get the predictors X and the labels y}
         \PY{n}{X}\PY{p}{,} \PY{n}{y} \PY{o}{=} \PY{n}{devnagari}\PY{o}{.}\PY{n}{get\PYZus{}data}\PY{p}{(}\PY{n}{target}\PY{o}{=}\PY{n}{devnagari}\PY{o}{.}\PY{n}{default\PYZus{}target\PYZus{}attribute}\PY{p}{)}\PY{p}{;} 
         \PY{n}{classes} \PY{o}{=} \PY{n}{devnagari}\PY{o}{.}\PY{n}{retrieve\PYZus{}class\PYZus{}labels}\PY{p}{(}\PY{n}{target\PYZus{}name}\PY{o}{=}\PY{l+s+s1}{\PYZsq{}}\PY{l+s+s1}{character}\PY{l+s+s1}{\PYZsq{}}\PY{p}{)} \PY{c+c1}{\PYZsh{} This one takes a while, skip if not needed}
\end{Verbatim}


    \begin{Verbatim}[commandchars=\\\{\}]
{\color{incolor}In [{\color{incolor}68}]:} \PY{k+kn}{from} \PY{n+nn}{random} \PY{k}{import} \PY{n}{randint}
         \PY{c+c1}{\PYZsh{} Take some random examples, reshape to a 32x32 image and plot}
         \PY{n}{fig}\PY{p}{,} \PY{n}{axes} \PY{o}{=} \PY{n}{plt}\PY{o}{.}\PY{n}{subplots}\PY{p}{(}\PY{l+m+mi}{1}\PY{p}{,} \PY{l+m+mi}{5}\PY{p}{,}  \PY{n}{figsize}\PY{o}{=}\PY{p}{(}\PY{l+m+mi}{10}\PY{p}{,} \PY{l+m+mi}{5}\PY{p}{)}\PY{p}{)}
         \PY{k}{for} \PY{n}{i} \PY{o+ow}{in} \PY{n+nb}{range}\PY{p}{(}\PY{l+m+mi}{5}\PY{p}{)}\PY{p}{:}
             \PY{n}{n} \PY{o}{=} \PY{n}{randint}\PY{p}{(}\PY{l+m+mi}{0}\PY{p}{,}\PY{l+m+mi}{90000}\PY{p}{)}
             \PY{n}{axes}\PY{p}{[}\PY{n}{i}\PY{p}{]}\PY{o}{.}\PY{n}{imshow}\PY{p}{(}\PY{n}{X}\PY{p}{[}\PY{n}{n}\PY{p}{]}\PY{o}{.}\PY{n}{reshape}\PY{p}{(}\PY{l+m+mi}{32}\PY{p}{,} \PY{l+m+mi}{32}\PY{p}{)}\PY{p}{,} \PY{n}{cmap}\PY{o}{=}\PY{n}{plt}\PY{o}{.}\PY{n}{cm}\PY{o}{.}\PY{n}{gray\PYZus{}r}\PY{p}{)}
             \PY{n}{axes}\PY{p}{[}\PY{n}{i}\PY{p}{]}\PY{o}{.}\PY{n}{set\PYZus{}xlabel}\PY{p}{(}\PY{l+s+s2}{\PYZdq{}}\PY{l+s+s2}{Class: }\PY{l+s+si}{\PYZpc{}s}\PY{l+s+s2}{\PYZdq{}} \PY{o}{\PYZpc{}} \PY{p}{(}\PY{n}{classes}\PY{p}{[}\PY{n}{y}\PY{p}{[}\PY{n}{n}\PY{p}{]}\PY{p}{]}\PY{p}{)}\PY{p}{)}
         \PY{n}{plt}\PY{o}{.}\PY{n}{show}\PY{p}{(}\PY{p}{)}\PY{p}{;}
\end{Verbatim}


    \begin{center}
    \adjustimage{max size={0.9\linewidth}{0.9\paperheight}}{output_12_0.pdf}
    \end{center}
    { \hspace*{\fill} \\}
    
    \begin{enumerate}
\def\labelenumi{\arabic{enumi}.}
\tightlist
\item
  Evaluate k-Nearest Neighbors, Logistic Regression and RandomForests
  with their default settings.

  \begin{itemize}
  \tightlist
  \item
    Take a stratified 10\% subsample of the data.
  \item
    Use the default train-test split and predictive accuracy. Is
    predictive accuracy a good scoring measure for this problem?
  \item
    Try to build the same models on increasingly large samples of the
    dataset (e.g. 10\%, 20\%,...). Plot the training time and the
    predictive performance for each. Stop when the training time becomes
    prohibitively large (this will be different for different models).
  \end{itemize}
\end{enumerate}

    2 . Optimize the value for the number of neighbors \(k\) (keep \(k\)
\textless{} 50) and the number of trees (keep \(n\_estimators\)
\textless{} 100) on the stratified 10\% subsample. - Use 10-fold
crossvalidation and plot \(k\) and \(n\_estimators\) against the
predictive accuracy. Which value of \(k\), \(n\_estimators\) should you
pick?

    3 . For the RandomForest, optimize both \(n\_estimators\) and
\(max\_features\) at the same time on the entire dataset. - Use a nested
cross-validation and a random search over the possible values, and
measure the accuracy. Explore how fine-grained this grid/random search
can be, given your computational resources. What is the optimal
performance you find? - Hint: choose a nested cross-validation that is
feasible. Don't use too many folds in the outer loop. - Repeat the grid
search and visualize the results as a plot (heatmap)
\(n\_estimators \times max\_features \rightarrow ACC\) with ACC
visualized as the color of the data point. Try to make the grid as fine
as possible. Interpret the results. Can you explain your observations?
What did you learn about tuning RandomForests?

    \subsection{3. Understanding Ensembles (5 points
(3+2))}\label{understanding-ensembles-5-points-32}

Do a deeper analysis of how RandomForests and Gradient Boosting reduce
their prediction error. We'll use the MAGIC telescope dataset
(http://www.openml.org/d/1120). When high-energy particles hit the
atmosphere, they produce chain reactions of other particles called
'showers', and you need to detect whether these are caused by gamma rays
or cosmic rays.

    \begin{Verbatim}[commandchars=\\\{\}]
{\color{incolor}In [{\color{incolor}4}]:} \PY{c+c1}{\PYZsh{} Get the data}
        \PY{n}{magic\PYZus{}data} \PY{o}{=} \PY{n}{oml}\PY{o}{.}\PY{n}{datasets}\PY{o}{.}\PY{n}{get\PYZus{}dataset}\PY{p}{(}\PY{l+m+mi}{1120}\PY{p}{)} \PY{c+c1}{\PYZsh{} Download MAGIC Telescope data}
        \PY{n}{X}\PY{p}{,} \PY{n}{y} \PY{o}{=} \PY{n}{magic\PYZus{}data}\PY{o}{.}\PY{n}{get\PYZus{}data}\PY{p}{(}\PY{n}{target}\PY{o}{=}\PY{n}{magic\PYZus{}data}\PY{o}{.}\PY{n}{default\PYZus{}target\PYZus{}attribute}\PY{p}{)}\PY{p}{;}
\end{Verbatim}


    \begin{Verbatim}[commandchars=\\\{\}]
{\color{incolor}In [{\color{incolor}6}]:} \PY{c+c1}{\PYZsh{} Quick visualization}
        \PY{n}{X}\PY{p}{,} \PY{n}{y}\PY{p}{,} \PY{n}{attribute\PYZus{}names} \PY{o}{=} \PY{n}{magic\PYZus{}data}\PY{o}{.}\PY{n}{get\PYZus{}data}\PY{p}{(}\PY{n}{target}\PY{o}{=}\PY{n}{magic\PYZus{}data}\PY{o}{.}\PY{n}{default\PYZus{}target\PYZus{}attribute}\PY{p}{,} \PY{n}{return\PYZus{}attribute\PYZus{}names}\PY{o}{=}\PY{k+kc}{True}\PY{p}{)}
        \PY{n}{magic} \PY{o}{=} \PY{n}{pd}\PY{o}{.}\PY{n}{DataFrame}\PY{p}{(}\PY{n}{X}\PY{p}{,} \PY{n}{columns}\PY{o}{=}\PY{n}{attribute\PYZus{}names}\PY{p}{)}
        \PY{n}{magic}\PY{o}{.}\PY{n}{plot}\PY{p}{(}\PY{n}{figsize}\PY{o}{=}\PY{p}{(}\PY{l+m+mi}{20}\PY{p}{,}\PY{l+m+mi}{10}\PY{p}{)}\PY{p}{)}
        \PY{c+c1}{\PYZsh{} Also plot the target: 1 = gamma, 0 = background}
        \PY{n}{pd}\PY{o}{.}\PY{n}{DataFrame}\PY{p}{(}\PY{n}{y}\PY{p}{)}\PY{o}{.}\PY{n}{plot}\PY{p}{(}\PY{n}{figsize}\PY{o}{=}\PY{p}{(}\PY{l+m+mi}{20}\PY{p}{,}\PY{l+m+mi}{1}\PY{p}{)}\PY{p}{)}\PY{p}{;}
\end{Verbatim}


    \begin{center}
    \adjustimage{max size={0.9\linewidth}{0.9\paperheight}}{output_18_0.pdf}
    \end{center}
    { \hspace*{\fill} \\}
    
    \begin{center}
    \adjustimage{max size={0.9\linewidth}{0.9\paperheight}}{output_18_1.pdf}
    \end{center}
    { \hspace*{\fill} \\}
    
    1 . Do a bias-variance analysis of both algorithms. For each, vary the
number of trees on a log scale from 1 to 1024, and plot the bias error
(squared), variance, and total error (in one plot per algorithm).
Interpret the results. Which error is highest for small ensembles, and
which reduced most by each algorithm as you use a larger ensemble? When
are both algorithms under- or overfitting? Provide a detailed
explanation of why random forests and gradient boosting behave this way.
- See lecture 3 for an example on how to do the bias-variance
decomposition - To save time, you can use a 10\% stratified subsample in
your initial experiments, but show the plots for the full dataset in
your report.

    2 . A \emph{validation curve} can help you understand when a model
starts under- or overfitting. It plots both training and test set error
as you change certain characteristics of your model, e.g. one or more
hyperparameters. Build validation curves for gradient boosting,
evaluated using AUROC, by varying the number of iterations between 1 and
500. In addition, use at least two values for the learning rate (e.g.
0.1 and 1), and tree depth (e.g. 1 and 4). This will yield at least 4
curves. Interpret the results and provide a clear explanation for the
results. When is the model over- or underfitting? Discuss the effect of
the different combinations learning rate and tree depth and provide a
clear explanation. - While scikit-learn has a \texttt{validation\_curve}
function, we'll use a modified version (below) that provides a lot more
detail and can be used to study more than one hyperparameter. You can
use a default train-test split.

    \begin{Verbatim}[commandchars=\\\{\}]
{\color{incolor}In [{\color{incolor}4}]:} \PY{c+c1}{\PYZsh{} Plots validation curves for every classifier in clfs. }
        \PY{c+c1}{\PYZsh{} Also indicates the optimal result by a vertical line}
        \PY{c+c1}{\PYZsh{} Uses 1\PYZhy{}AUROC, so lower is better}
        \PY{k}{def} \PY{n+nf}{validation\PYZus{}curve}\PY{p}{(}\PY{n}{clfs}\PY{p}{,} \PY{n}{X\PYZus{}test}\PY{p}{,} \PY{n}{y\PYZus{}test}\PY{p}{,} \PY{n}{X\PYZus{}train}\PY{p}{,} \PY{n}{y\PYZus{}train}\PY{p}{)}\PY{p}{:}
            \PY{k}{for} \PY{n}{n}\PY{p}{,}\PY{n}{clf} \PY{o+ow}{in} \PY{n+nb}{enumerate}\PY{p}{(}\PY{n}{clfs}\PY{p}{)}\PY{p}{:}
                \PY{n}{test\PYZus{}score} \PY{o}{=} \PY{n}{np}\PY{o}{.}\PY{n}{empty}\PY{p}{(}\PY{n+nb}{len}\PY{p}{(}\PY{n}{clf}\PY{o}{.}\PY{n}{estimators\PYZus{}}\PY{p}{)}\PY{p}{)}
                \PY{n}{train\PYZus{}score} \PY{o}{=} \PY{n}{np}\PY{o}{.}\PY{n}{empty}\PY{p}{(}\PY{n+nb}{len}\PY{p}{(}\PY{n}{clf}\PY{o}{.}\PY{n}{estimators\PYZus{}}\PY{p}{)}\PY{p}{)}
        
                \PY{k}{for} \PY{n}{i}\PY{p}{,} \PY{n}{pred} \PY{o+ow}{in} \PY{n+nb}{enumerate}\PY{p}{(}\PY{n}{clf}\PY{o}{.}\PY{n}{staged\PYZus{}decision\PYZus{}function}\PY{p}{(}\PY{n}{X\PYZus{}test}\PY{p}{)}\PY{p}{)}\PY{p}{:}
                    \PY{n}{test\PYZus{}score}\PY{p}{[}\PY{n}{i}\PY{p}{]} \PY{o}{=} \PY{l+m+mi}{1}\PY{o}{\PYZhy{}}\PY{n}{roc\PYZus{}auc\PYZus{}score}\PY{p}{(}\PY{n}{y\PYZus{}test}\PY{p}{,} \PY{n}{pred}\PY{p}{)}
        
                \PY{k}{for} \PY{n}{i}\PY{p}{,} \PY{n}{pred} \PY{o+ow}{in} \PY{n+nb}{enumerate}\PY{p}{(}\PY{n}{clf}\PY{o}{.}\PY{n}{staged\PYZus{}decision\PYZus{}function}\PY{p}{(}\PY{n}{X\PYZus{}train}\PY{p}{)}\PY{p}{)}\PY{p}{:}
                    \PY{n}{train\PYZus{}score}\PY{p}{[}\PY{n}{i}\PY{p}{]} \PY{o}{=} \PY{l+m+mi}{1}\PY{o}{\PYZhy{}}\PY{n}{roc\PYZus{}auc\PYZus{}score}\PY{p}{(}\PY{n}{y\PYZus{}train}\PY{p}{,} \PY{n}{pred}\PY{p}{)}
        
                \PY{n}{best\PYZus{}iter} \PY{o}{=} \PY{n}{np}\PY{o}{.}\PY{n}{argmin}\PY{p}{(}\PY{n}{test\PYZus{}score}\PY{p}{)}
                \PY{n}{learn} \PY{o}{=} \PY{n}{clf}\PY{o}{.}\PY{n}{get\PYZus{}params}\PY{p}{(}\PY{p}{)}\PY{p}{[}\PY{l+s+s1}{\PYZsq{}}\PY{l+s+s1}{learning\PYZus{}rate}\PY{l+s+s1}{\PYZsq{}}\PY{p}{]}
                \PY{n}{depth} \PY{o}{=} \PY{n}{clf}\PY{o}{.}\PY{n}{get\PYZus{}params}\PY{p}{(}\PY{p}{)}\PY{p}{[}\PY{l+s+s1}{\PYZsq{}}\PY{l+s+s1}{max\PYZus{}depth}\PY{l+s+s1}{\PYZsq{}}\PY{p}{]}
                \PY{n}{test\PYZus{}line} \PY{o}{=} \PY{n}{plt}\PY{o}{.}\PY{n}{plot}\PY{p}{(}\PY{n}{test\PYZus{}score}\PY{p}{,}
                                     \PY{n}{label}\PY{o}{=}\PY{l+s+s1}{\PYZsq{}}\PY{l+s+s1}{learn=}\PY{l+s+si}{\PYZpc{}.1f}\PY{l+s+s1}{ depth=}\PY{l+s+si}{\PYZpc{}i}\PY{l+s+s1}{ (}\PY{l+s+si}{\PYZpc{}.2f}\PY{l+s+s1}{)}\PY{l+s+s1}{\PYZsq{}}\PY{o}{\PYZpc{}}\PY{p}{(}\PY{n}{learn}\PY{p}{,}\PY{n}{depth}\PY{p}{,}
                                                                         \PY{n}{test\PYZus{}score}\PY{p}{[}\PY{n}{best\PYZus{}iter}\PY{p}{]}\PY{p}{)}\PY{p}{)}
        
                \PY{n}{colour} \PY{o}{=} \PY{n}{test\PYZus{}line}\PY{p}{[}\PY{o}{\PYZhy{}}\PY{l+m+mi}{1}\PY{p}{]}\PY{o}{.}\PY{n}{get\PYZus{}color}\PY{p}{(}\PY{p}{)}
                \PY{n}{plt}\PY{o}{.}\PY{n}{plot}\PY{p}{(}\PY{n}{train\PYZus{}score}\PY{p}{,} \PY{l+s+s1}{\PYZsq{}}\PY{l+s+s1}{\PYZhy{}\PYZhy{}}\PY{l+s+s1}{\PYZsq{}}\PY{p}{,} \PY{n}{color}\PY{o}{=}\PY{n}{colour}\PY{p}{)}
                
                \PY{n}{plt}\PY{o}{.}\PY{n}{xlabel}\PY{p}{(}\PY{l+s+s2}{\PYZdq{}}\PY{l+s+s2}{Number of boosting iterations}\PY{l+s+s2}{\PYZdq{}}\PY{p}{)}
                \PY{n}{plt}\PY{o}{.}\PY{n}{ylabel}\PY{p}{(}\PY{l+s+s2}{\PYZdq{}}\PY{l+s+s2}{1 \PYZhy{} area under ROC}\PY{l+s+s2}{\PYZdq{}}\PY{p}{)}
                \PY{n}{plt}\PY{o}{.}\PY{n}{axvline}\PY{p}{(}\PY{n}{x}\PY{o}{=}\PY{n}{best\PYZus{}iter}\PY{p}{,} \PY{n}{color}\PY{o}{=}\PY{n}{colour}\PY{p}{)}
                
            \PY{n}{plt}\PY{o}{.}\PY{n}{legend}\PY{p}{(}\PY{n}{loc}\PY{o}{=}\PY{l+s+s1}{\PYZsq{}}\PY{l+s+s1}{best}\PY{l+s+s1}{\PYZsq{}}\PY{p}{)}
\end{Verbatim}



    % Add a bibliography block to the postdoc
    
    
    
    \end{document}
